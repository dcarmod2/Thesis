\documentclass[10pt]{amsart}
\usepackage{amsmath,amsthm,amssymb,amsfonts,amsopn,mathrsfs,mathtools}
\usepackage[margin=1in]{geometry}
\usepackage{tikz}
\usetikzlibrary{arrows}
\usepackage[all]{xy}
\usepackage{hyperref}
\usepackage{color}
\usepackage{tabulary}
\usepackage{cite}
\usepackage[normalem]{ulem}
\usepackage{aliascnt}
\usepackage{todonotes}
\setlength{\marginparwidth}{2cm}
\hypersetup{colorlinks=true,linktoc=all,linkcolor=blue}
\usepackage{kpfonts}





\newcommand{\Z}{\mathbb Z}

\newcommand{\mbb}{\mathbb}
\newcommand{\mc}{\mathcal}
\newcommand{\dual}{\,\check{}}
\newcommand{\RGr}{\mathbb R\mathrm{Gr}}
\newcommand{\hRGr}{\mathbb R\mathscr{G}{\mathrm{r}}}
\newcommand{\Set}{\mathbf{Set}}
\newcommand{\Sm}[1]{\mathbf{Sm}_{#1}}
\newcommand{\Gpd}{\mathbf{Gpd}}
\newcommand{\BigVB}[1]{\mathbf{BigVB}({#1})}
\newcommand{\Cat}{\mathbf{Cat}}
\newcommand{\Iso}{\mathrm{Iso}}
\newcommand{\St}{\mathrm{St}}
\newcommand{\Span}{\mathrm{Span}}


\DeclareMathOperator{\Spec}{Spec}
\DeclareMathOperator*{\holim}{holim}
\DeclareMathOperator*{\hocolim}{hocolim}
\DeclareMathOperator*{\colim}{colim}
\DeclareMathOperator*{\im}{im}
\DeclareMathOperator*{\coker}{coker}
\DeclareMathOperator{\Ob}{Ob}
\DeclareMathOperator{\Hom}{Hom}
\DeclareMathOperator*{\Mor}{Mor}
\DeclareMathOperator{\PreSm}{Pre_{Sm_S^{C_2}}}
\DeclareMathOperator{\rk}{rk}
\DeclareMathOperator{\tr}{tr}
\DeclareMathOperator{\can}{can}
\DeclareMathOperator{\iHom}{\mathbf{Hom}}
%%%%%%%%%%%%%% Diagrams

\newcommand{\commsq}[4]{\xymatrix{{#1} \ar[r]\ar[d] & {#2} \ar[d] \\
    {#3} \ar[r] & {#4}}}

%%%%%%%%%%%%%% Theorem Environments
\theoremstyle{plain}

\newtheorem{lemma}{Lemma}
\newtheorem{theorem}{Theorem}
\newtheorem{proposition}[lemma]{Proposition}
\theoremstyle{definition}
\newtheorem{corollary}[lemma]{Corollary}
\newtheorem{example}[lemma]{Example}
\newtheorem{definition}[lemma]{Definition}
\newtheorem{notation}[lemma]{Notation}

\newtheorem{remark}[lemma]{Remark}
\newtheorem{question}{Question}
\newtheorem{conjecture}{Conjecture}
\newtheorem{construction}{Construction}


\begin{document}

\tableofcontents

\section{Introduction}

\section{Equivariant Topologies}

\todo{Add things about the equivariant Nisnevich topology}
\begin{notation}
Throughout this section, $G$ will be either a finite group or the
group scheme over $S$ associated to a finite group. Recall that to pass between
finite groups and group schemes over $S$, we form the scheme
$\coprod_G S$ with multiplication (using that fiber products commute
with coproducts in $Sch/S$):
\[
\coprod_G S \times_S \coprod_G S \cong \coprod_{(g_1,g_2)\in G \times
  G} S \xrightarrow{\mu} \coprod_G S. 
\]

Whenever we write down a pullback square involving schemes, we'll
tacitly be thinking of $G$ as a group scheme, and $X \times Y$ will
really mean $X \times_S Y$.
\end{notation}


\begin{definition}
For a $G$-scheme $X$, the isotropy group scheme is a group scheme
$G_X$ over $X$ defined by the cartesian square
\[
\xymatrix{G_X \ar[r] \ar[d] & G \times X \ar[d]^{(\mu_X,id_X)} \\ X
  \ar[r]^{\Delta_X} & X \times X}.
\]
\end{definition}

\begin{definition}
Let $X$ be a $G$-scheme. The scheme-theoretic stabilizer of a point
$x$ in $X$ is the pullback
\[
\xymatrix{G_x \ar[r] \ar[d] & G_X \ar[d]\\ \Spec k(x) \ar[r] & X.}
\]

By the pasting lemma, this is the same as the pullback
\[
\xymatrix{G_x \ar[r] \ar[d] & G \times X \ar[d]\\ \Spec k(x) \ar[r] &
  X \times X.}
\]
\end{definition}

\begin{definition}
Let $X$ be a $G$-scheme, and define the \emph{set-theoretic}
stabilizer $S_x$ of $x \in X$ to be $\{g \in G
\mid gx = x\}$ where we think of $G$ as a finite group. 
\end{definition}

\begin{remark}
With notation as above, the underlying set of the scheme-theoretic
stabilizer $G_x$ can be described as
\[
G_x = \{g \in S_x \mid \text{ the induced morphism $g : k(x)
  \rightarrow k(x)$ equals } id_{k(x)}\}.
\]
\end{remark}



The example below shows that set-theoretic and scheme-theoretic
stabilizers need not agree.

\begin{example} (Herrmann \cite{GrpSchHerr})
Let $k$ be a field, and consider the $k$-scheme given by a finite
Galois extension $k \hookrightarrow L$. Let $G = Gal(L/k)$ be the
Galois group. The set-theoretic stabilizer of the unique point in
$\Spec L$ is $G$ itself, while the scheme-theoretic stabilzer is
$\{e\} \subset G$. 
\end{example}

\begin{remark}
Recall that if $Z \rightarrow X$ is a monomorphism of schemes, then
the forgetful functor from schemes to sets preserves any pullback $Z
\times_X Y$. The canonical examples of monomorphisms in schemes are
closed embeddings, open immersions, and maps induced by localization. 

Recall as well that the forgetful functor $GSch/S \rightarrow Sch/S$
is a right adjoint, hence preserves pullbacks.

Since the inclusion of a point $\Spec k(x) \hookrightarrow X \times_S X$
will be a closed embedding for any separated scheme, the difference
between the set-theoretic and scheme-theoretic stabilizers is given by
the fact that the underlying space of $X \times_S X$ is not
necessarily the fiber product of the underlying spaces. Indeed, in the
example above, $\Spec L \times_k \Spec L \cong \coprod_{g \in G} \Spec
k$, whereas the pullback in spaces is just a single point. 
\end{remark}

\subsection{The equivariant \'Etale topology}

\begin{notation}
Let $S$ be a  $G$-scheme. The equivariant \'etale topology on $Sm_S$
will denote the site whose covers are \'etale covers whose component
morphisms are equivariant.
\end{notation}

\begin{definition} (Thomason)
An equivariant map $f : Y \rightarrow X$ is said to be
\emph{isovariant} if it induces an isomorphism $G_Y \cong G_X \times_X
Y$. A collection $\{f_i : X_i \rightarrow X\}_{i \in I}$ of
equivariant maps called an isovariant etale cover if it is an
equivariant etale cover such that each $f_i$ is isovariant. 
\end{definition}

\begin{remark}
The isovariant topology is equivalent to the topology whose covers are
equivariant, stabilizer preserving, \'etale maps. We'll use this
notion more often in computations. 
\end{remark}

\begin{remark}
The points in the isovariant \'etale topology are schemes of the form
$G \times^{G_x} \Spec(\mc O^h_{X,\overline x})$ where $\overline x
\rightarrow x \rightarrow X$ is a geometric point, and $(-)^h$ denotes
strict henselization. 
\end{remark}

\begin{remark}
If $G = C_2$, then $G_x = \{e\}$ or $G_x = C_2$ for all $x \in X$. If
$G_x = \{e\}$, then $G \times^{G_x} \Spec(\mc O^h_{X,\overline x})
\cong C_2 \times \Spec(\mc O^h_{X,\overline x}) \cong \Spec(\mc
O^h_{X,\overline x})\coprod \Spec(\mc O^h_{X,\overline x})$ with a
free action. If $G_x = C_2$, then $G \times^{G_x} \Spec(\mc
O^h_{X,\overline x}) = \Spec(\mc
O^h_{X,\overline x})$.
\end{remark}

The following example shows that there are equivariant \'etale covers
which are not isovariant:

\begin{example}
Fix a scheme $X$ with trivial $C_2$-action, and consider the scheme $X \coprod X$ with the
switch action. The map $X \coprod X \rightarrow X$ is an equivariant
\'etale cover, but it is not stabilizer preserving. Indeed, the switch
action on $X \coprod X$ is free, and the set-theoretic (hence
scheme-theoretic) stabilizers are all trivial. On the other hand, the
scheme theoretic stabilizers of the trivial action are all $C_2$.
\end{example}


% \begin{lemma} (Sketchy...)
% Fix a scheme $X$ with $G$-action and consider the quotient stack $S = [X/G]$. The
% \'etale site on $Sm_S$ is equivalent to the isovariant \'etale site on $Sm^G_X$.
% \end{lemma}

% \begin{proof} (Idea only)


% An object $Y \rightarrow S$ in $Sm_S$ consists of a $Y$-torsor $T
% \rightarrow Y$ and an equivariant map $T \rightarrow X$. Upon taking
% quotients, we get a (hopefully still \'etale, need to check Deligne's notes) map $Y \cong T/G \rightarrow X/G$, where $X/G$ is
% an algebraic space. By a theorem of Thomason, the \'etale site over
% the algebraic space $X/G$ is equivalent to the isovariant \'etale site
% on $Sm^G_X$. Thus given an \'etale map $Y \rightarrow X/G$, we get an
% isovariant \'etale map $Y \rightarrow X$. From this we get a diagram
% $Y \leftarrow Y \times G \rightarrow X$ representing a map $Y
% \rightarrow S$.
% \end{proof}


\begin{lemma}
Fix a ring $R$, and fix an ideal $I \subset R$, $J \subset R[x]$. Let
$B = R[x]/J$. Then $B/IB \cong (R/I)[x]/\overline J$.
\end{lemma}

\begin{proof}
First, recall that $R[x]/IR[x] \cong (R/I)[x]$ by the obvious map
reducing the coefficients of a polynomial. Then $B/IB \cong
R[x]/(IR[x]+J) \cong (R[x]/IR[x])/J \cong (R/I)[x]/\overline J$.
\end{proof}

\begin{example}
Let $R$ be a commutative ring with 2 invertible and involution $- : R
\rightarrow R$. Let $a \in R^\times$. Then $\Spec
R[\sqrt{a},\sqrt{\overline a}] \rightarrow \Spec R$ is an equivariant \'etale cover. 
\end{example}

\begin{proof}
First, note that if $a$ has a square root in $R$, so does $\overline
a$, and the result is trivial. Assume that this is not the case. Give
the ring $R[\sqrt{a},\sqrt{\overline a}]$ an action by $r_0 +
r_1\sqrt{a} + r_2\sqrt{\overline a} \mapsto \overline r_0 + \overline
r_1\sqrt{\overline a} + \overline r_2 \sqrt{a}$. The map $R
\rightarrow R[\sqrt{a},\sqrt{\overline a}]$ is clearly equivariant, so
we need only check that it's an \'etale cover.

First, note that it is indeed a cover: because
$R[\sqrt{a},\sqrt{\overline a}]$ is a module-finite extension of $R$
(hence integral),
surjectivity after taking $\Spec$ follows from the injectivity of the
map of rings by the lying over property for integral extensions. 

Now, we claim that the map is \'etale. We'll prove that it's the
composite of two \'etale maps, $R \rightarrow R[\sqrt{a}] \rightarrow
R[\sqrt{a},\sqrt{\overline a}]$. Since $\overline a$ must also be a
unit, it's enough to show that $R \rightarrow R[\sqrt{a}]$ is \'etale. It's clearly flat because
$R[\sqrt{a}]$ is a free module over $R$, so we just
have to check that it's unramified. Let $B =
R[\sqrt{a}]$. Fix a maximal ideal $m \subset
B$, and let $I = R \cap m$. By the lemma above,
\[
\frac{B}{IB} \cong (R/I)[x]/(x^2 - a) \cong (R/I)[\sqrt{a}].
\]

Now if $a \neq 0$ in $R/I$, then $x^2-a$ will be a separable polynomial. But because $a$ is a unit, it's not contained in any prime
ideal, and hence not contained in $I$.

An easy consequence of the going up theorem (recall that we have
an integral extension), is that $I$ is a maximal ideal in $R$; hence,
$(R/I)[\sqrt{a},\sqrt{\overline a}]$ is a finite separable field
extension of $R/I$. Since localization commutes with taking quotients,
it follows that the map is unramified. 
\end{proof}

\begin{example}
A similar argument shows that $\Spec R[\sqrt{a}] \coprod \Spec
R[\sqrt{\overline a}] \rightarrow \Spec R$ is an equivariant \'etale cover.
\end{example}

\begin{lemma}
With notation as above, assume that $a$ is a fixed point of the
involution $- : R \rightarrow R$. There's an induced action on $ R[\sqrt{a}]$
which fixes $\sqrt{a}$, and the map $\Spec R[\sqrt{a}] \rightarrow
\Spec R$ is stabilizer preserving w.r.t. this action.
\end{lemma}

\begin{proof}
Let $p \subset
R[\sqrt{a}]$ be a prime ideal such that $\overline p = p$. Let $g$ denote the
non-trivial element of $C_2$. The induced map on stalks is (by abuse
of notation) the
inclusion $f : k(p \cap R) \hookrightarrow k(p \cap R)[\sqrt{a}]$. By equivariance,
we have a commutative diagram
\[
\xymatrix{k(p \cap R) \ar[r]^f \ar[d]^{\widetilde g} & k(p \cap R)[\sqrt{a}] \ar[d]^g \\
  k(p \cap R) \ar[r]^f & k(p)[\sqrt{a}].}
\]

Now if $g$ induces the identity map $k(p \cap R) \rightarrow k(p \cap
R)$, and hence is an element of the scheme-theoretic stabilizer, then
$\widetilde g$ is an element of $Gal(k(p \cap R)[\sqrt{a}]/k(p \cap
R)$. In other words, $\widetilde g$ is either the identity map, or is
the map which sends $\sqrt{a} \rightarrow -\sqrt{a}$. By construction,
the involution on $R[\sqrt{a}]$ sends $\sqrt{a} \mapsto \sqrt{a}$, so
that $G_p = G_{f(p)}$.

If $g$ doesn't fix $k(p \cap R)$, then since $f$ is a monomorphism, clearly $\widetilde g$ can't
fix $k(p \cap R)[\sqrt{a}]$, and again we have $G_p = G_{f(p)}$. It
follows that $f$ is an isovariant map. 
\end{proof}





% For the next lemma, we'll use the following result of Rydh

% \begin{proposition} (Rydh)
% Let $f : X \rightarrow Y$ be an equivariant \'etale map. The subset
% $X_0 \subseteq X$ of points at which $f$ is stabilizer preserving is
% an invariant open subset. 
% \end{proposition}

\begin{lemma}
With notation as above, say $a - 
\overline a \in R^*$. The equivariant \'etale cover $f : \Spec R[\sqrt{a}] \coprod \Spec
R[\sqrt{\overline a}] \rightarrow \Spec R$ is stabilizer preserving.
\end{lemma}

\begin{proof}
The action on $ \Spec R[\sqrt{a}] \coprod \Spec
R[\sqrt{\overline a}] \rightarrow \Spec R$ is free, so that all the
set-theoretic (and hence scheme-theoretic) stabilizers are trivial. 

The assumption that $a - \overline a$ is not in any prime ideal implies that
if $p$ is a fixed point of the involution, $i : R_p/pR_p \rightarrow
R_p/pR_p$ is not the identity map, so that the scheme-theoretic
stabilizers of the action on $\Spec R$ are all trivial.
\end{proof}

\begin{example}
Even if $a$ is a unit, it's certainly not true in general that
$ a - \overline a \in R^*$. Consider the ring $R = \Z[t,t^{-1}]$ with
involution given by $t \mapsto -t$. Then $t - \overline t = 2t \not
\in R^*$. Furthermore, $(2t) = (2)$ is a prime ideal in $R$ fixed by
the involution. It's contained in the maximal ideal $(2,t-1)$. Note
that this ideal is also fixed by the involution: $t - 1 \mapsto -t-1$
and $-t-1 = 1-t -2 \in (2,t-1)$. The residue field at this maximal
ideal is $\Z/2$. The only nonzero ring map of this field is the
identity, so that the scheme-theoretic stabilizer of $(2,t-1)$ is
$C_2$.

Note that if we wanted an example for a ring with 2 invertible, we
could take $R=\Z[\frac{1}{2}][t,t^{-1}]$, and consider the element
$\frac{3}{2}t$ and the maximal ideal $(3,t-\frac{3}{2})$. One also has
to note that the induced map on the residue field $\Z/3$ is the
identity, which follows simply because the involution is unital (and
because it gives a well-defined map on the residue field!).
\end{example}

\subsection{The equivariant Nisnevich topology}

Similarly to the non-equivariant case, the equivariant Nisnevich
topology is defined a particularly nice cd-structure. While there are
a few different definitions of this topology in the literature which
can give non-Quillen equivalent model structures, we use the
definition from \cite{GrpSchHell}.



\begin{definition}
A distinguished equivariant Nisnevich square is a cartesian square in
$\Sm{S}^G$
\[
\xymatrix{B \ar[r]\ar[d] & Y \ar[d]^p \\ A \ar@{^{(}->}[r]^i & X}
\]
where $j$ is an open immersion, $p$ is \'etale, and
$(Y-B)_{\mathrm{red}} \rightarrow (X-A)_{\mathrm{red}}$ is an
isomorphism. 
\end{definition}

\begin{definition}
The equivariant Nisnevich $cd$-structure on $\Sm{S}^G$ is the
collection of distinguished equivariant Nisnevich squares in $\Sm{S}^G$.
\end{definition}

\begin{remark}
For finite groups $G$, any smooth $G$-scheme is Nisnevich-locally
affine. 
\end{remark}

\subsection{Computations with Equivariant Spheres}

Because we'll be using equivariant spheres to index our spectra, we'll
record some of their basic properties here. Though there are exotic
elements of the Picard group even in non-equivariant stable motivic
homotopy theory, we'll be concerned with the four building blocks
$S^1, S^\sigma = \colim ((C_2)_+ \rightarrow S^0), \mbb G_m, \mbb
G_m^{\sigma}$.

\begin{lemma}
Let $\mbb P^\sigma$ denote $\mbb P^1$ with the action defined by $[x:y] \mapsto
[y:x]$. There is an equivariant Nisnevich square
\[
\xymatrix{C_2\times \mbb G_m^\sigma \ar[r]\ar[d]^{\pi_2} & \mbb P^1 -
  \{0\} \coprod \mbb P^1 - \{\infty\} \ar[d]^f \\ G_m^\sigma \ar[r]^i & \mbb P^\sigma}
\]
\end{lemma} 

\begin{proof}
Here, we identify $\mbb G_m^\sigma$ with $\mbb P^\sigma -
\{0,\infty\}$. The map $i$ is clearly an open immersion. Its
complement is $\mbb \{0,\infty\}$, and $f$ maps
$\pi^{-1}(\{0,\infty\})$ isomorphically onto
$\{0,\infty\}$. Furthermore, $f$ is a disjoint union of open
immersions, and hence is (in particular) etale. 
\end{proof}

\begin{lemma}
The following square is a homotopy pushout square:

\[
\xymatrix{(C_2)_+\wedge \mbb (G_m^\sigma)_+ \ar[r]\ar[d]^{\pi_2} &
  (C_2)_+ \ar[d]^f \\ (G_m^\sigma)_+ \ar[r]^i & \mbb P_+^1}
\]
\end{lemma}

\begin{proof}
The above square is equivalent to the square

\[
\xymatrix{(C_2)_+\wedge \mbb (G_m^\sigma)_+ \ar[r]\ar[d]^{\pi_2} &
  (C_2)_+ \wedge \mbb A^1_+ \ar[d]^f \\ (G_m^\sigma)_+ \ar[r]^i & \mbb P_+^1}.
\] 

By the lemma above, 

\[
\xymatrix{(C_2\times \mbb G_m^\sigma)_+ \ar[r]\ar[d]^{\pi_2} & (C_2
  \times \mbb A^1)_+\ar[d]^f \\ (G_m^\sigma)_+ \ar[r]^i & \mbb P_+^1}
\]
is a homotopy pushout square. But adding a disjoint basepoint is a
monoidal functor, so $X_+ \wedge Y_+ \cong (X \times Y)_+$ and this
square is equivalent to the desired square.
\end{proof}

\begin{lemma}
$\mbb P^\sigma \approx S^\sigma \wedge \mbb G_m^\sigma$.
\end{lemma}

\begin{proof}
Let $Q$ denote the homotopy cofiber of $(C_2 \times \mbb G_m^\sigma)_+
\rightarrow (\mbb G_m^\sigma)_+$, and $\widetilde Q$ denote the homotopy
cofiber of $(C_2 \times \mbb A^1)_+ \rightarrow \mbb P^\sigma_+$. Then the
lemma above implies that $Q \approx \widetilde Q$. 

$Q$ is the homotopy cofiber of $
(C_2)_+ \wedge \mbb (G_m^\sigma)_+ \rightarrow S^0 \wedge \mbb
(G_m^\sigma)_+$, which is just $S^\sigma \wedge \mbb
(G_m^\sigma)_+$. Recall that $\colim(* \leftarrow X \rightarrow X
\wedge Y_+) \cong X \wedge Y$ since this is $X \wedge \colim(*
\leftarrow S^0 \rightarrow Y_+)$. Thus the cofiber of $S^\sigma
\rightarrow Q$ is $S^\sigma \wedge \mbb G_m^\sigma$.

The diagram below in which the horizontal rows are cofiber sequences
\[
\xymatrix{(C_2)_+ \ar[r]\ar[d]^{id} & S^0 \ar[d] \ar[r]& S^\sigma
  \ar[d] \\(C_2)_+ \ar[r]\ar[d] & \ar[r]\ar[d] \mbb P^\sigma_+ & \widetilde
  Q \ar[d] \\ \star \ar[r] & \mbb P^\sigma \ar[r] & T }
\]

implies that the cofiber of $S^\sigma \rightarrow \widetilde Q$ is
$\mbb P^\sigma$. 

The result now follows from the commutativity of the following diagram
and homotopy invariance of homotopy cofiber:
\[
\xymatrix{S^\sigma \ar[r]^{id}\ar[d] & S^\sigma \ar[d] \\ Q \ar[r]^{\sim} \ar[d]&
\widetilde Q \ar[d]\\ S^\sigma \wedge \mbb G_m^\sigma \ar[r] & \mbb P^\sigma}.
\]

\end{proof}

\section{Hermitian Forms on Schemes}

\subsection{Definitions}

\begin{definition}
Let $R$ be a ring with involution $- : R \rightarrow R$. A \emph{hermitian
module over $R$} is a finitely generated projective module-$R$, $M$, together
with a map
\[
b : M \otimes_{\Z} M \rightarrow R
\]  

such that, for all $a \in R$,
\begin{enumerate}
\item $b(xa,y) = \overline a b(x,y)$,
\item $b(x,ya) = b(x,y) a$,
\item $b(x,y) = \overline{b(y,x)}$.
\end{enumerate}
\end{definition}

\begin{definition}
Let $R$ be a ring with involution $-$. Given a right $R$-module $M$, define
a left $R$-module, denoted $\overline M$ as follows: $\overline M$ has the same underlying abelian
group as $M$, and the action is given by $r \cdot m = m \cdot \overline r$. If $R$ is
commutative, we can define an $R$-bimodule by $m \cdot r =
m \overline r$ and $r \cdot m = m \overline r$. 
\end{definition}

\begin{remark}
Let $R$ be a commutative ring. Given an involution $\sigma : R \rightarrow R$, and an $R-R$-bimodule
$M$ as above, we can identify $\overline M$ with $\sigma_*M$. Indeed,
$\sigma_*M$ is an $R-R$-bimodule via the rule $r \cdot \overline m =
\sigma(r)\overline M$, and since $R$ is commutative, we can view this
either as a left or right $R$-module. 
\end{remark}

\begin{remark}
Another way to define a Hermitian form over a ring $R$ with involution
$\sigma$ is to give a finitely generated projective mod-$R$ $M$ together
with an $R-R$-bimodule map  
\[
b :  M \otimes_{\Z}  M \rightarrow R
\]
where we view $R$ as a
bimodule over itself just by $r_1 \cdot r \cdot r_2 = r_1rr_2$, $M$ as
a left $R$-module via the involution, and such that $b(x,y) = \sigma(b(y,x))$. If we remove the final
condition, we obtain a sesquilinear form. 
\end{remark}

By the usual duality, we have a third definition:

\begin{definition}
A
hermitian module over a ring $R$ with involution is a finitely
generated projective $R$-module $M$ together with an $R$-linear map $b : M
\rightarrow \overline{M}\dual = M^*$ such that $b = b^*can_M$, where
$b^* : M^{**} \rightarrow M^*$ is given by $(b(f))(m) = f(b(m))$.
\end{definition}

Now, we generalize the above definitions to schemes.

\begin{definition}
Let $X$ be a scheme, and $M$ a quasi-coherent (locally of finite presentation) $\mc O_X$-module. Define $\mc O_X\dual
= \uline{Hom}(M,\mc O_X)$.
\end{definition}

\begin{definition}
Let $X$ be a scheme with involution $\sigma$, and $M$ a right $\mc
O_X$-module. Note that there's an induced map $\sigma^{\#} : \mc O_X
\rightarrow \sigma_*\mc O_X$. Define the right (note that we're
working with sheaves of commutative rings, so we can do this) $\mc O_X$-module $\overline
M$ to be $\sigma_*M$ with $\mc O_X$ action induced by the map
$\sigma^{\#}$. That is, if $m \in \sigma_*M(U)$, and $c \in \mc
O_X(U)$, then $m \cdot c = m \cdot \sigma^{\#}(c)$. Note that this
last product is defined, because $m \in \sigma_*M(U) =
M(\sigma^{-1}(U))$, $c \in \sigma_*\mc O_X(U) = \mc
O_X(\sigma^{-1}(U))$, and $M$ is a right $\mc O_X$-module. 
\end{definition}



\begin{remark}
We have two choices for the definition of the dual $M^*$. We can
either define $M^* = Hom_{mod-\mc O_X}(\sigma_*M,\mc O_X)$, or we can
define $M^* = \sigma_*Hom_{mod-\mc O_X}(M,\mc O_X)$. We claim that
these two choices of dual are naturally isomorphic. 
\end{remark}



\begin{proof}
Let $f : \sigma_*M|_U \rightarrow \mc O_X|_U$ be a map of right $\mc
O_X|_U$-modules. Post-composing with the map $\mc O_X|_U \rightarrow
\sigma_* \mc O_X|_U$ yields a map $\overline f : \sigma_*M|_U
\rightarrow \sigma_*\mc O_X|_U$, a.k.a. a map $M|_{\sigma^{-1}U}
\rightarrow \mc O_X|_{\sigma^{-1}U}$. Note that $\sigma_*Hom_{mod-\mc
  O_X}(M,\mc O_X)(U) = Hom_{mod-\mc
  O_X}(M,\mc O_X)(\sigma^{-1}U)$, so that $\overline f \in \sigma_*Hom_{mod-\mc
  O_X}(M,\mc O_X)(U)$.

On the other hand, given $g \in \sigma_*Hom_{mod-\mc O_X}(M,\mc
O_X)(U)$, so that $g : \sigma_*M|_U
\rightarrow \sigma_*\mc O_X|_U$, we can postcompose with
$\sigma_*(\sigma^\#)$ to get a map $\widetilde g : \sigma_*M|_*
\rightarrow \sigma_*\sigma_*\mc O_X|_U = \mc O_X|_U$. Since $\sigma^2
= id$, this is clearly the inverse to the map above. 

It's clear that these assignments are natural, since they're just
postcomposition with a natural transformation. 
\end{proof}

\begin{definition}
Define the adjoint module $M^*$ to be $Hom_{mod-\mc O_X}(\sigma_*M,\mc
O_X)$. By the remark above, it doesn't really matter which of the two
possible definitions we choose here. 
\end{definition}


\begin{definition}
Given a right $\mc O_X$-module $M$, we define the double dual
isomorphism $\can : M \rightarrow M^{**}$ as follows: given an open $U
\subseteq X$, we define a map
\[
M(U) \rightarrow Nat(\sigma_* Nat(\sigma_* M,\mc O_X)|_U, \mc O_X|_U)
= Nat(Nat(\sigma_* M|_{\sigma(U)}, \mc O_X|_{\sigma(U)}),\mc O_X|_U)
\]
by $u \mapsto \eta_u$, where for an open $V \subseteq U$,
\[
(\eta_u)_V(\gamma) = (\sigma^\#)_{V}^{-1}(\gamma_{\sigma(V)}(u|_V)).
\]

Here $\gamma \in Nat(\sigma_* M|_{\sigma(U)}, \mc
O_X|_{\sigma(U)})$ and $\sigma^\#$ is the morphism of sheaves
$\sigma^\# : \mc O_X \rightarrow \sigma_* \mc O_X$. Note that $\gamma_{\sigma(V)}(u|_V)$ makes sense
because $\sigma_*M(\sigma(V)) = M(V)$.

More globally, there's an evaluation map
\[
ev_\sigma : M \otimes \sigma_*Nat(\sigma_*M,\mc O_X) \rightarrow \mc O_X
\]
defined by the composition
\[
M \otimes \sigma_*Nat(\sigma_*M,\mc O_X) \cong M \otimes
Nat(M,\sigma_*\mc O_X) \xrightarrow{ ev} \sigma_* \mc O_X
\xrightarrow{\sigma^\#} \mc O_X
\]
which under adjunction yields the above map. 
\end{definition}

\begin{definition}
Let $X$ be a scheme with involution $- : X \rightarrow X$. A
\emph{hermitian vector bundle over $X$} is a locally free right $\mc
O_X$-module $V$ with an $\mc O_X$-module map $V \rightarrow V^*$.
\end{definition}

\begin{remark}
  Recall that there's an equivalence of categories between locally
  free coherent sheaves on $X$ and geometric vector bundles given by
  $M \mapsto \mathbf{Spec}Sym(M\dual)$ in one direction and the sheaf
  of sections in the other. For locally free sheaves, we have $M\dual
  \otimes N\dual \cong (M \otimes N)\dual$ so that the functor is
  monoidal. We will use this to think of a hermitian
  form as a map of schemes $V \otimes V \rightarrow \mbb A^1$.
\end{remark}

Below we give the key example of a hermitian vector bundle.

\begin{example}

Define (diagonal) hyperbolic $n$-space over a scheme $(S,-)$ with involution to be $\mbb A^{2n}_S$
with the hermitian form $(x_1,\dots,x_{2n},y_1,\dots,y_{2n}) \mapsto
\sum_{i=1}^n  \overline x_{2i-1} y_{2i-1} -  \overline x_{2i}
y_{2i}$. Denote this hermitian form by $h_{\mathrm{diag}}$.

As defined this way, the matrix of this hermitian form is
\[
\begin{bmatrix}
1 & 0 & \cdots \\
0 & -1 & 0 & \cdots \\
\vdots & \vdots & \vdots \\
0 & \cdots & \cdots & \cdots & -1
\end{bmatrix}
\]
the diagonal matrix diag$(1,-1,1,\dots,-1)$. For this definition to
give a hermitian space isometric to other standard definitions of the
hyperbolic form, it's crucial that 2 be invertible. 

The isometries of $\mbb H_{\mbb R}$ (where we give it the hyperbolic
form above) have the form
\[
\begin{bmatrix}
a & b\\
\pm b & \pm a
\end{bmatrix}
\]
with $a = \pm\sqrt{1 + b^2}, b \in \mbb R$ (or $a^2 - b^2 = 1$). The
usual identification with $\mbb R^\times \rtimes C_2$ follows by
considering the decomposition $a^2 - b^2 = 1 \iff (a+b)(a-b) = 1$.
\end{example}

\begin{example}
Similarly to above, we can define a hyperbolic form $h$ by the matrix
\[
\begin{bmatrix}
0 & I\\
I & 0
\end{bmatrix}.
\]

This form is isometric to the above form, and we'll use both forms below.
\end{example}

\subsubsection{Properties}

\begin{lemma}
Given a map of schemes with involution $f : (Y,i_Y) \rightarrow (X,i_X)$
and a (non-degenerate) hermitian vector bundle $(V,\omega)$ on $X$, $f^*(V)$ is
a (non-degenerate) hermitian
vector bundle on $Y$.
\end{lemma}


\begin{proof}
The pullback of a locally free $\mc O_X$-module is a locally free $\mc
O_Y$-module, so we just need to check that it's hermitian. Given the
map $\omega : V \rightarrow V^*$, we get an induced map $f^*V
\rightarrow f^*(V^*)$ which is an isomorphism if $\omega$ is. Thus we
just need to check that $f^*(V^*) \cong (f^*V)^*$. But pullback
commutes with sheaf dual for locally free sheaves of finite rank, so
we just need to check that changing the module structure via the
involution commutes with pullback; that is, we need to check that
$f^*(\overline V) = \overline{f^*(V)}$. However, this is clear since
the structure map on $f^*(\overline V)$ is given by

%Recall that pullback commutes with product because
%product of sheaves of abelian groups agrees with direct sum,
%and pullback (of sheaves of abelian groups) is a left adjoint.
\[
O_Y \times f^*V \cong f^* \mc O_X \times f^*V \xrightarrow{f^*(-) \times id} f^*\mc(O_X) \times f^*(V)
\rightarrow f^*(V).
\]
\end{proof}

\begin{theorem} (Knus \cite{HermKnus} 6.2.4)
Let $(M,b)$ be an $\epsilon$-hermitian space over a division ring
$D$. Then $(M,b)$ has an orthogonal basis in the following cases:
\begin{enumerate}
\item the involution of $D$ is not trivial
\item the involution of $D$ is trivial, the form is symmetric, and
  char $D \neq 2$.
\end{enumerate}
\end{theorem}

\begin{lemma} (Knus)
Let $(M,b)$ be a hermitian module, and $(U,b|_U)$ be a non-degenerate
f.g. projective Hermitian submodule. Then $M = U \oplus U^\perp$.
\end{lemma}

\begin{proof}
Since $b|_U : U \rightarrow U^*$ is an isomorphism, given an $m \in
M$, there exists $u \in U$ s.t. $b(m,-)|_U = b(u,-)|_U$. But then
$b(m-u,-)|_U = 0$, so that $m-u \in U^\perp$, and $m = u + m-u$. Thus
$M = U + U^\perp$. Since $\phi|_U$ is non-degenerate, $U \cap U^\perp
= 0$, so we're done. 
\end{proof}

\subsection{Hermitian Forms on Semilocal Rings}

\begin{theorem}
Let $R$ be a ring, and let $E$ be a hermitian module over
$R$. Let $I \subset Jac(R)$ be an ideal. For every orthogonal decomposition $\overline E = \overline F
\perp \overline G$ of $\overline E = E/IE$ over $R/I$, where $\overline
F$ is a free non-singular subspace of $\overline E$, there exists an
orthogonal decomposition $E = F \perp G$ of $E$ with $F$ free and
non-singular, and $F/IF = \overline F, G/IG = \overline G$.
\end{theorem}

\begin{proof}
Write $\overline F = \langle \overline x_1 \rangle \oplus \dots \oplus
\langle \overline x_n \rangle$ with $\overline x_i \in \overline F$
and $\det(\overline b(\overline x_i,\overline x_j)) \in (R/I)^*$. Choose
representatives $x_i \in E$ of $\overline x_i$, and let $F = Rx_1 +
\dots + Rx_n$. We claim that the $x_i$ are independent, so that $F$ is
free: indeed, if $\lambda_1x_1 + \dots + \lambda_nx_n = 0$, then we
get $n$ equations $\lambda_1b(x_1,x_i) + \dots + \lambda_nb(x_n,x_i) =
0$. But we know that $\det(b(x_i,x_j)) = t\in R^*$, since $1-st \in I$
for some $s$ by assumption, but then $st$ cannot be contained in any
maximal ideal, so $st \in R^*
\implies t \in R^*$. It follows that the $\lambda_i$ are zero, so that
the $x_i$ are independent as desired. The determinant fact also shows
that $F$ is regular, so by the lemma above, it has an orthogonal
summand $G$. By construction $F/I = \overline F$, so that $\overline G
= (\overline
F)^{\perp} = (F/I)^{\perp} = F^{\perp}/I = G/I$.
\end{proof}

\begin{lemma}
Hermitian forms over $R_1 \times R_2$ (with trivial involution) are in bijection with $Herm(R_1)
\times Herm(R_2)$.
\end{lemma}

\begin{proof}
First, recall that modules over $R_1 \times R_2$ correspond to a
module over $R_1$ and a module over $R_2$. Indeed, consider the
standard idempotents $(1,0) = e_1, (0,1) = e_2$. Fix a module $M$ over
$R_1 \times R_2$. Then $M = e_1M \oplus e_2M$. Indeed, any $m \in M$
can be written as $e_1m + e_2m = (e_1+e_2)m = m$. Furthermore, if
$e_1m_1 = e_2m_2$, then $e_2e_1m_1=e_2e_2m_2 \implies 0 = e_2m_2$. 

Now, a hermitian form $M \otimes M \rightarrow R_1 \times R_2$ is
determined by two maps $M \otimes M \rightarrow R_1$ and $M \otimes M
\rightarrow R_2$. Writing $M = e_1M \oplus e_2M$, we note that, by
linearity, it must be the case that $e_1M \otimes e_2M \rightarrow R_1
\times R_2$ is the zero map; to wit, $b(e_1m_1,e_2m_2) =
e_1e_2b(m_1,m_2) = 0$. Thus this hermitian form is determined
completely by the maps $e_1M \otimes e_1M \rightarrow R_1 \times R_2$
and $e_2M \otimes e_2M \rightarrow R_1 \times R_2$. Finally, note
that, again by linearity, we see that $e_1M \otimes e_1M \rightarrow
R_2$ is the zero map: $b(e_1m_1,e_1m_2) = b(e_1^2m_1,e_1m_2) =
e_1b(e_1m_1,e_1m_2)$, and $e_1R_2 = 0$. Similarly for the other
map. Hence, at the end of the day, the hermitian form is completely
determined by the maps $e_1M \otimes e_1M \rightarrow R_1$ and $e_2M
\otimes e_2M \rightarrow R_2$. 
\end{proof}

\begin{corollary}
Free hermitian modules diagonalize over rings with finitely many maximal
ideals (semi-local rings).
\end{corollary}

\begin{proof}
By the Chinese Remainder Theorem, $R/(m_1 \cap \dots \cap m_n) \cong
R/m_1 \times \cdots \times R/m_n = F_1 \times \dots \times F_n$. We
claim that Hermitian forms over finite products of fields
diagonalize, and then the result will follow from the above theorem. By induction and the lemma above, a hermitian module $M$
is determined by hermitian modules $M_i$ over $F_i$, $i =
1,\dots,n$ as $M = M_1 \oplus M_2 \oplus \cdots \oplus M_n$ with
action $(f_1,\dots,f_n) \cdot (m_1,\cdots,m_n) =
(f_1m_1,\dots,f_nm_n)$. Each $M_i$ can be diagonalized into $M_i = \langle a_{1,i}
\rangle \perp \dots \perp \langle a_{m,i}\rangle$ (it's important to
note here that since $M$ is free, the rank of each $M_i$ is the same). Thus a
diagonalization of $M$ is given by $\langle (a_{1,1},\dots,a_{1,n})
\rangle \perp \dots \perp \langle (a_{1,m},
\dots,a_{m,n})\rangle$. 
\end{proof}

Now, let $R$ be a ring with involution, and $I \subseteq Jac(R)$ and
ideal. Then $C_2\cdot I \subseteq Jac(R)$ is an ideal fixed by the
involution. 

The following corollary has the same proof as the theorem above, the
only subtlety is that we need the quotient ring to inherit the
involution to make sense of an induced hermitian module.

\begin{corollary}
Let $R$ be a ring with involution, and let $E$ be a hermitian module over
$R$. Let $I \subset Jac(R)$ be an ideal fixed by the involution. For every orthogonal decomposition $\overline E = \overline F
\perp \overline G$ of $\overline E = E/IE$ over $R/I$, where $\overline
F$ is a free non-singular subspace of $\overline E$, there exists an
orthogonal decomposition $E = F \perp G$ of $E$ with $F$ free and
non-singular, and $F/IF = \overline F, G/IG = \overline G$.
\end{corollary}

\begin{corollary}
Let $R$ be a local ring with involution (necessarily a map of local
rings). Then any Hermitian module (which is necessarily free)
over $R$ diagonalizes. 
\end{corollary}

\begin{lemma} 
Let $R$ be a ring, and consider the ring $R \times R$ with the
involution that switches factors. Then any module $M$ can be
written as $e_1M \oplus e_2M$ as above. A non-degenerate hermitian form on this
module is determined by a map $e_1M \otimes e_2M \rightarrow R \times
R$, i.e. as a matrix it has the form
\[
\begin{bmatrix}
0 & A \\
\overline{A}^t & 0
\end{bmatrix}.
\]

where $A$ is invertible. 
\end{lemma}

\begin{proof}
The first claim is just that $b(e_1x,e_1y) = 0 = b(e_2x,e_2y)$ for any
$x,y \in M$. This follows because $b(e_1x,e_1y) = b(e_1^2x,e_1^2y) =
\overline{e_1}e_1b(e_1x,e_1y) = e_2e_1b(e_1x,e_1y) = 0$. Similarly for
$b(e_2x,e_2y)$. The statement about the matrix follows by identifying
the map $M \otimes \overline M \rightarrow R \times R$ with an isomorphism $M
\rightarrow \overline M^*$ and using the direct sum decomposition. 
\end{proof}

\begin{corollary} \label{HermFormsOverSwitch} 
Let $R$ be as in the lemma additionally with 2 invertible. Then $M \cong H(e_1M)$, where $H$ denotes
the hyperbolic space functor. 
\end{corollary}

\begin{proof}
The assumption that 2 is invertible implies that $M$ is an even
hermitian space in the notation of Knus. Now by the corollary above
$b|_{e_1M} = 0$, so $M$ has a direct summand such that $e_1M =
e_1M^{\perp}$. Now corollary 3.7.3 in Knus applies to finish the proof.
\end{proof}

\begin{corollary}
Let $R$ be a semi-local ring with involution. Then any hermitian
module over $R$ diagonalizes.
\end{corollary}

\begin{proof}
Using the theorem above and reducing modulo the Jacobson radical
(which is always stable under the involution), it
suffices to prove the corollary for $R$ a finite product of
fields. Then $R = F_1 \times \dots \times F_n$ is semi-simple, and
hence we can index the fields in a particularly nice way (proof is by
considering idempotents), writing $R = A_1 \times \dots \times A_m \times B_1 \times
\dots B_{n-m}$ such that $A_i$ is fixed by the involution, and
$\sigma(B_{2i}) = B_{2i+1}$, $\sigma(B_{2i+1}) = B_{2i}$. Now, any
finitely generated module $M$ can be written as a direct sum $M =
\bigoplus_{i=1}^m M_i \bigoplus_{i=1}^{\frac{n-m}{2}} N_{2i} \oplus
N_{2i-1}$. By the two lemmas above, the form when restricted to each
$M_i$ or $N_{2i} \oplus N_{2i-1}$ is diagonalizable, so the form is
diagonalizable (see the proof of the non-involution case).   
\end{proof}


\begin{corollary}
Hermitian vector bundles are locally hyperbolic in the isovariant \'etale topology.
\end{corollary}

\begin{proof}
The points in the isovariant \'etale topology are either strictly
henselian local rings with
trivial involution or a product of two such rings with switch
involution. If the ring is a local ring, the fact that all non-degenerate hermitian
forms are trivial is well-known, since we have square roots. If the
ring is hyperbolic, then all non-degenerate hermitian forms over the
ring are hyperbolic by the corollary above.
\end{proof}


\begin{corollary}
Let $R \times R$ be a ring with the involution which switches
factors. Fix a hermitian module $M$ over $R$, and let $N = e_1M$ (see
above for notation). Then
$O(M) \cong GL(N)$.
\end{corollary}

\begin{proof}
In corollary \ref{HermFormsOverSwitch} above, we identified
non-degenerate hermitian
forms over such rings as hyperbolic. Thus it suffices to prove the
statement for forms of the form 
\[
\begin{pmatrix}
0 & 1\\
can & 0
\end{pmatrix}
\]
\end{proof}

\begin{lemma}
Let $(R,i)$ be a ring with involution with 2 invertible, and let
$(M,b)$ be a non-degenerate hermitian module over $R$. There exists an equivariant
\'etale cover $\{U_i \rightarrow \Spec R\}$ of $\Spec R$ such that
$(M,b)|_{U_i}$ is trivial.
\end{lemma}

\begin{proof}
For a fixed prime $p$, consider the semilocal ring $R_{(p)} \times
R_{i(p)}$. By the universal property of localization, there's an induced involution $i$ on  $R_{(p)} \times
R_{i(p)}$ given by $(f_1,f_2) \mapsto (i(f_2),i(f_1))$. The
restriction of $M$ to this ring has a diagonalization $v(p)^*bv(p) =
D$. Choose a greatest common denominator $(f_1,f_2)$ for the entries of $v(p)^*$
and $v(p)$. By finding a common denominator and inverting the
determinant, there's an element $(g_1,g_2) \in R-p_1 \times R-p_2$
s.t. $v(p)^*bv(p) = D$ is an equality in $R[g_1^{-1},i(g_2)^{-1}] \times
R[i(g_1)^{-1},g_2^{-1}]$. By construction, the set of $g$ s.t. we
have such a diagonalization is not contained in any maximal
ideal. Thus there exist $(g_1,g_2),\dots,(g_{n-1}.g_n)$ s.t. $b$
diagonalizes over $R[g_i^{-1},i(g_{i+1})^{-1}] \times
R[i(g_i)^{-1},g_{i+1}^{-1}]$ and s.t. $\prod R[g_i^{-1},i(g_{i+1})^{-1}] \times
R[i(g_i)^{-1},g_{i+1}^{-1}] \rightarrow R$ is an equivariant Zariski
cover. Now by adjoining square roots of the units corresponding to the
diagonalization in each $R[g_i^{-1},i(g_{i+1})^{-1}] \times
R[i(g_i)^{-1},g_{i+1}^{-1}]$ (and their images under the
involution), if necessary, we obtain an \'etale cover $E_1 \times
\dots \times E_n$ of $\Spec R$ s.t. $(M,b)$ is trivial when pulled
back to each $E_i$. 
\end{proof}

\begin{lemma}\label{lem:perp_pullback}
Let $(V,\phi)$ be a non-degenerate hermitian vector bundle over a
scheme with trivial involution $X$, and let $(M,\phi|_M)$ be a (possibly degenerate)
sub-bundle. Given a map of schemes $g : Y \rightarrow X$, there is a
canonical isomorphism
$g^*(M^\perp) \cong (g^*M)^\perp$.
\end{lemma}

\begin{proof}
Recall that, by definition, $M^\perp = \ker(V \xrightarrow{\phi} V^*
\rightarrow M^*)$. Equivalently, $M^\perp$
is defined by the exact sequence
\[
0 \rightarrow M^\perp \rightarrow V \rightarrow M^* \rightarrow 0.
\] 

It follows that the composite map $g^*(M^\perp) \rightarrow g^*V
\rightarrow g^*(M^*)$ is zero, and hence by universal property of kernel there's a canonical map
\[
g^*(M^\perp) \rightarrow \ker(g^*V \rightarrow g^*(M^*) \cong (g^*(M))^*) = (g^*(M))^\perp
\]
 where we've used the canonical isomorphism $g^*(M^*) \cong
 (g^*(M))^*$ for locally free sheaves.

We claim that this map is an isomorphism. It suffices to check on
stalks, where the map can be identified with a map 
\[
M^\perp_{g(y)} \otimes \mc O_{Y,y} \rightarrow \ker(V_{g(y)} \otimes
\mc O_{Y,y} \rightarrow M^*_{g(y)} \otimes \mc O_{Y,y}).
\]

But $V_{g(y)} \cong M^\perp_{g(y)} \oplus M^*_{g(y)}$, so the sequence
\[
0 \rightarrow M^\perp_{g(y)} \otimes \mc O_{Y,y}\rightarrow V_{g(y)}
\otimes \mc O_{Y,y} \rightarrow M^*_{g(y)} \otimes \mc O_{Y,y}
\rightarrow 0
\]
is split exact, and the canonical map is an isomorphism. 
\end{proof}



\section{Hermitian Grassmannians}

Fix a separated base scheme $S$ with trivial involution. The goal of
this section is to define a sheaf on $\Sm{S}^{C_2}$, denoted $\RGr_V$, which represents
non-degenerate sub-bundles of a given hermitian vector bundle $V$. 

\subsection{The definition of $\RGr$}

\begin{lemma}\label{lem:action_presheaf}
Let $\mc F$ be a presheaf on $\Sm{S}$ and let $a : \mc F \implies
\mc F$ be a natural transformation s.t. $a \circ a = id_{\mc F}$. Then
there's an associated presheaf on $\Sm{S}^{C_2}$ defined by the formula
$(X,\sigma : X \rightarrow X) \mapsto \mc F(X)^{C_2}$ where the action
of $C_2$ on $\mc F(X)$ is defined by $f \mapsto  a_X\mc F(\sigma)(f)$.
\end{lemma}

\begin{proof}
Note that this is indeed a $C_2$-action, since $a_X \mc F(\sigma)(
a_X\mc F(\sigma)(f)) = \mc F(\sigma) a_X(a_X \mc F(\sigma)(f)) = \mc
F(\sigma)(\mc F(\sigma)(f)) = f$ using naturality. 
\end{proof}

Fix a (possibly degenerate) hermitian vector bundle $(V,\phi)$ over the base scheme $S$
(which has trivial involution). 

We'll define a presheaf $\RGr : (\Sm{S}^{C_2})^{op} \rightarrow \Set$
by first defining a presheaf on $\Sm{S}$, showing that it's
representable, equipping with an action, then taking the corresponding
representable functor on $\Sm{S}^{C_2}$.

\begin{itemize}
\item On objects, $\RGr(V)(f: X \rightarrow S)$ for an $S$-scheme $f :
  X \rightarrow S$ is a split surjection $(p,s)$
\[
\xymatrix{f^*V \ar@{>>}[r]_p & \ar@/_1.0pc/@{-->}[l]_sW},
\]

where $W$ is locally free. 

Here by an isomorphism of split surjections we mean a diagram
\[
\xymatrix{f^*V \ar@{>>}[r]_p \ar@{=}[d]& \ar@/_1.0pc/@{-->}[l]_sW \ar[d]^\phi\\
f^*V \ar@{>>}[r]_{p'} & \ar@/_1.0pc/@{-->}[l]_{s'}W'}
\]

such that $\phi$ is an isomorphism satisfying $\phi \circ p = p'$ and $s = s'\circ \phi$.

\item Given a morphism 
\[
\xymatrix{Y\ar[dr]_h \ar[rr]^g &&X\ar[dl]^f\\& S&}
\]
over $S$, define
\[
\RGr_V(g)(\xymatrix{f^*V \ar@{>>}[r]_p & \ar@/_1.0pc/@{-->}[l]^sW}) = 
\xymatrix{h^*V \ar[r]^{can} & g^*f^*V \ar@{>>}[r]_{g^*p} &
  \ar@/_1.0pc/@{-->}[l]_{g^*s} g^*W}.
\]
\end{itemize}
There's a natural action of $C_2$, on $\RGr_V$, whose non-trivial
natural transformation will be denoted $\eta$. Define $\eta$ as
follows: 

 Fix an object $X \in \Sm{S}$. Define
\[
\eta_X(\xymatrix{f^*V \ar@{>>}[r]_p & \ar@/_1.0pc/@{-->}[l]_sW}) =
\xymatrix{f^*V \ar@{>>}[r]_q & \ar@/_1.0pc/@{-->}[l]_t(\ker p)^\perp}.
\]

We claim that this is well-defined. % First, the map
% \[
% \xymatrix{f^*V \ar@{>>}[r]_p & \ar@/_1.0pc/@{-->}[l]_sW} \mapsto \xymatrix{f^*V \ar@{>>}[r]_p & \ar@/_1.0pc/@{-->}[l]_s\ker(p)}
% \]
% is well defined by the splitting lemma for abelian categories. We'll show that the $\perp$
% map is also well-defined, and $\eta_X$ is the composite of these maps.

Recall that 
\[
W^\perp = \ker(f^*V \xrightarrow{f^*\phi} f^*(V^*) \xrightarrow{can}
(f^*V)^* \xrightarrow{s^*} W^*).
\]



Leaving out the $can$ map for convenience, we get a split exact sequence 
\[
\xymatrix{0 \ar[r] & W^\perp \ar[r] & f^*V \ar[r]_{s^*}& \ar@/_1.0pc/[l]_{p^*} W^*
\ar[r] & 0}.
\]

By the splitting lemma for abelian categories, $f^*V \cong W^\perp
\oplus W^*$, and hence there's a split surjection $f^*V
\twoheadrightarrow W^\perp$ with $W^\perp$ locally free. 

Given an isomorphism 

\[
\xymatrix{f^*V \ar@{>>}[r]_p \ar@{=}[d]& \ar@/_1.0pc/@{-->}[l]_sW \ar[d]^\psi\\
f^*V \ar@{>>}[r]_{p'} & \ar@/_1.0pc/@{-->}[l]_{s'}W'}
\]
we get an isomorphism of (split) diagrams
\[
\xymatrix{f^*V \ar[r]^{f^*\phi}\ar@{=}[d] &(f^*V)^*\ar[r]^{s^*} \ar@{=}[d] & W^* \ar[d]^{(\psi^{-1})^*}\\
f^*V \ar[r]^{f^*\phi} & (f^*V)^* \ar[r]_{(s')^* }& (W')^*}
\]
and hence an isomorphism of split surjections
\[
\xymatrix{f^*V \ar@{>>}[r]_q \ar@{=}[d]& \ar@/_1.0pc/@{-->}[l]_tW^\perp \ar[d]^\delta\\
f^*V \ar@{>>}[r]_{q'} & \ar@/_1.0pc/@{-->}[l]_{t'}(W')^\perp},
\]

so that $\eta_X$ is a well-defined map of sets. Given a map of schemes
$g : Y \rightarrow X$, such that $f\circ g = h$ and an element 
\[
\xymatrix{f^*V \ar@{>>}[r]_p & \ar@/_1.0pc/@{-->}[l]_sW}
\]

in $\RGr_V(X)$, 
\begin{align*}
\RGr(g) \circ \eta_X(\xymatrix{f^*V \ar@{>>}[r]_p &
  \ar@/_1.0pc/@{-->}[l]_sW}) = & \RGr(g)( \xymatrix{f^*V \ar@{>>}[r]_q
  & \ar@/_1.0pc/@{-->}[l]_t(\ker p)^\perp})\\
 = & \xymatrix{h^*V \ar[r]^{can} & g^*f^*V \ar@{>>}[r]_{g^*q} &
  \ar@/_1.0pc/@{-->}[l]_{g^*t} g^*((\ker(p))^\perp)}
\end{align*}

while
\begin{align*}
\eta_Y\circ\RGr(g)(\xymatrix{f^*V \ar@{>>}[r]_p &
  \ar@/_1.0pc/@{-->}[l]_sW})) = \xymatrix{h^*V \ar[r]^{can} & g^*f^*V \ar@{>>}[r]_{q'} &
  \ar@/_1.0pc/@{-->}[l]_{t'} (g^*(\ker(p)))^\perp}
\end{align*}

By Lemma \ref{lem:perp_pullback}, there's a canonical isomorphism
$g^*((\ker(p)^\perp)) \rightarrow (g^*(\ker(p)))^\perp$, and under
this isomorphism $q'$ and $t'$ correspond to $g^*q$, and $g^*t$,
respectively. This concludes the check of naturality.


Now by Lemma \ref{lem:action_presheaf}, there's a presheaf $\RGr :
\Sm{S}^{C_2} \rightarrow \Set$. To determine its values on a
$C_2$-scheme $(X,\sigma)$, we note that a fixed point of the action of Lemma
\ref{lem:action_presheaf} is determined by an isomorphism of split
surjections

\[
\xymatrix{f^*V \ar@{>>}[r]_q \ar@{=}[d]& \ar@/_1.0pc/@{-->}[l]_t\sigma^*(\ker(p)^\perp) \ar[d]^\psi\\
f^*V \ar@{>>}[r]_{p} & \ar@/_1.0pc/@{-->}[l]_{s}\ker(p)}
\]

Note that because $\sigma$ is an involution, for any $\mc O_X$-module
$M$, there's a canonical isomorphism of $\mc O_X$-modules $\sigma_*M \cong
\sigma^*M$. Thus there's a natural isomorphism 
\[
\Hom_{mod-\mc
  O_X}(\sigma_*f^*V,-) \cong \Hom_{mod-\mc O_X}(\sigma^*f^*V,-) \cong
\Hom_{mod-\mc O_X}(f^*V,-).
\]

It follows that any Hermitian form
\[
\phi : f^*V \rightarrow \Hom_{mod-\mc O_X}(f^*V,\mc O_X)
\]
can be promoted to a Hermitian form
\[
\widetilde\phi : f^*V \rightarrow \Hom_{mod-\mc O_X}(\sigma_*f^*V,\mc O_X)
\]
compatible with an involution $\sigma$ on $X$.

Let $(M,\phi|_M)$ be a hermitian sub-bundle of $f^*V$ over the scheme $X$ with
trivial involution. We claim that $\sigma^*(M^\perp)$ is the
orthogonal complement of $M$ viewed as a hermitian sub-bundle of
$f^*V$ with the promoted form $\widetilde \phi$. Said differently, we
claim that
\[
\sigma^*(\ker(f^*V \xrightarrow{\phi|_M} \Hom(M,\mc O_X)) \cong \ker(f^*V
\xrightarrow{\widetilde \phi|_M} \Hom(\sigma_*M,\mc O_X)).
\]

But using the natural isomorphism between $\sigma^*$ and $\sigma_*$,
together with the natural isomorphisms $\sigma^*\Hom(M,\mc O_x) \cong
\Hom(M,\mc O_X)$ and $\sigma^*f^*V \cong f^*V$, this becomes a question
of whether $\sigma^*$ is left exact. In general it isn't, but because
it's naturally isomorphic to $\sigma_*$, and $\sigma_*$ is left exact, the claim
follows. 

% First note that $\sigma^{-1}M(U) = M(\sigma(U))$. Thus
% $\sigma^*M(U) = M(\sigma(U)) \otimes_{\mc O_X(\sigma(U))} \mc O_X(U)$
% with module structure via the map on the right. On the other hand,  $\sigma_*M(U) = M(\sigma^{-1}(U)) = M(\sigma(U))$,


\subsection{Representability of $\RGr$}


Fix a hermitian vector bundle $(V,\phi)$ over $S$ where $\dim(V) = n$
and $S$ is a scheme with
trivial involution. Then the underlying scheme of $\RGr(V)$ is the
pullback 
\[
\xymatrix{\RGr(V) \ar[r]\ar[d] & \uline{\Hom}_{\mc O_S}(V,V) \times
  \uline{\Hom}_{\mc O_S}(V,V) \ar[d]^{\circ, id} \\ \uline{\Hom}_{\mc
    O_S}(V,V) \ar[r]^-\Delta & \uline{\Hom}_{\mc O_S}(V,V) \times \uline{\Hom}_{\mc O_S}(V,V)}
\]

where the right vertical map sends $p \mapsto (p \circ p, p)$. In
other words, the underlying scheme is the scheme of idempotent
endomorphisms of $V$. The action corresponds to the map $p \mapsto
p^\dagger$, where $p^\dagger$ is the adjoint of $p$ with respect to
the form $\phi$. 

Note that using this description, an equivariant map $(X,\sigma) \rightarrow
\RGr(V)$ corresponds to an idempotent $p : V_X \rightarrow V_X$ such
that $\phi^{-1}(\gamma^{-1}(\sigma^*p)\gamma)^*\phi = p$, where we're being cavalier
and using $\ast$ to denote both dual (on the outside) and pullback (by
$\sigma$). Here $\gamma$ is the canonical isomorphism $V_X
\xrightarrow{\gamma} \sigma^*V_X$; if the structure map of $X$ is $f :
X \rightarrow S$, then $\gamma$ arises from the equality $\sigma \circ f = f$.

Note that the form on $V_{(X,\sigma)}$ is by definition the composite

\[
\widetilde \phi : V_X \xrightarrow{\phi} V_X^* \xrightarrow{(\gamma^*)^{-1}} \sigma^*V_X^*
\xrightarrow{(\eta^*)^{-1}} \sigma_*V_X^*,
\]
and the adjoint of $p$ is given by $\widetilde \phi^{-1} (\sigma_*p)^*
\widetilde \phi$. Expanding, this is

\[
\phi^{-1}(\gamma^*)(\eta^*)(\eta^*)^{-1}(\sigma^*p)^*(\eta^*)(\eta^*)^{-1}(\gamma^*)^{-1}\phi
= \phi^{-1}(\gamma^{-1}(\sigma^*p)\gamma)^*\phi,
\]

and so we recover the condition that $p^\dagger  =  p$, which
corresponds to the fact that $V_X = \ker p \perp \im p$, and hence the
restriction of the form on $V_X$ to $\im p$ (and $\ker p$) is
non-degenerate. 

To summarize, the underlying scheme of $\RGr(V)$ represents
idempotents, and equivariant maps pick out those idempotents which
correspond to orthogonal projections. 

\begin{remark}

Now fix a dimension $d$ and a non-degenerate hermitian vector bundle
$(V,\phi)$ over $S$.  Recalling that the trace of an idempotent
coincides with the rank of the image, define $\RGr_d(V)$ to be the closed
subscheme of $\RGr(V)$ cut out by $\tr(p) = d$, where $\tr$ is the
trace of an endomorphism. In other words, $\RGr_d(V)$ is the pullback
\[
\xymatrix{\RGr_d(V) \ar[r]\ar[d] & \RGr(V) \ar[d]^{\tr} \\ \{d\}
  \ar[r] & \Z}
\]

where $\Z$ is the locally constant sheaf on $\Sm{S}^{C_2}$. The
requirement that $V$ be non-degenerate is necessary so that the action
on $\RGr(V)$ sends rank $d$ subspaces to rank $d$ subspaces and hence
induces an action on $\RGr_d(V)$.
\end{remark}

\subsection{The universal idempotent}

Denote by $g : \RGr_d(V) \rightarrow S$ the structure map of
$\RGr_d(V)$.  Because $\RGr_d(V)$ is representable by a $C_2$-scheme, there's an
idempotent $g^*(V) \rightarrow g^*(V)$ corresponding to the identity map $id: \RGr_d(V) \rightarrow
\RGr_d(V)$. This idempotent is simply the idempotent which over a
point of $\RGr_d(V)$ represented by an idempotent $p : V \rightarrow
V$ restricts to $p$. There's an action $\sigma$ on $\RGr_d(V) \times_S V$
induced by the action on $\RGr_d(V)$, and using the fact that $\sigma
p \sigma = p^\dagger$ one can see that this idempotent is
non-degenerate with respect to the promoted hermitian form on $g^*(V)$
compatible with the involution on $\RGr_d(V)$.

\begin{remark}
Since we've shown that $\RGr(V)$ represents non-degenerate hermitian
subbundles of $V$, at this point we'll move away from explicitly
referring to split surjections and just represent the sections of
$\RGr(V)$ by non-degenerate subbundles. 
\end{remark}

\begin{definition}
Let $\mbb H_S$ denote the hyperbolic plane. For $V \in \mbb H^\infty$
a constant rank non-degenerate subbundle,
let $|V|$ denote the rank of $V$. Order such subbundles of $\mbb
H^\infty$ by inclusion, and denote the resulting poset $P$. Given an inclusion $V \hookrightarrow V'$
of non-degenerate subbundles, denote by $V'-V$ the complement of $V$
in $V'$. Let $\mc H : P \rightarrow \mathrm{Fun}(\Sm{S}^{C_2,
  op},Set)$ be the functor which on objects sends a subbundle $V$ to $
\RGr_{|V|} (V \perp \mbb H^\infty)$. Given an inclusion $V \hookrightarrow V'$, the induced map
$\RGr_{|V|}(V \perp \mbb H^\infty) \rightarrow \RGr_{|V'|}(V' \perp
\mbb H^\infty)$ is given by $E \mapsto E \perp (V'-V)$. Note that
because $V$ is non-degenerate, $V \perp (V'-V) = V'$. Define
\[
\RGr_\infty = \colim \mc H.
\]

\end{definition}



% We claim that $\RGr(\mc O^{n}_S,\phi)$ is representable by a
% $C_2$-scheme over $S$. To prove this, it suffices to show that the
% underlying functor without the natural action is
% representable. Following the usual strategy for showing the
% Grassmannan is representable, let $I$ be the set of subsets of
% $\{1,\dots,n\}$. For fixed $i$ with $|i| = n-k$, there's a unique
% order preserving map
% $\overline i : \{1,\dots,n-k\} \rightarrow \{1,\dots,n\}$ with image $i$. Let
% \[
% s_{i,k} : \mc O^{n-k} \rightarrow \mc O^n
% \]
% be the inclusion represented by the matrix $[s_{kt}]$ where $s_{kt} =
% \delta_{k\overline i(t)}$. 

% Let $F_{i,k}(f: X \rightarrow S)$ be the collection of split surjections
% \[
% \xymatrix{f^*\mc O^n_S \ar@{>>}[r]_p & \ar@/_1.0pc/@{-->}[l]_sW}
% \]

% such that $p\circ s_i$ is surjective. 

% We claim that $F_i$ is representable. Because $p \circ s_i$ is a surjection
% between locally free modules of the same rank, it's an
% isomorphism. Assume WLOG $i = \{1,\dots,n-k\}$. Then using $\psi$ to
% identify $W$ with $f^*\mc O^n_S$, the diagram above is a split surjection
% \[
% \xymatrix{f^*\mc O^n_S \ar@{>>}[r]_{(p\circ s_i)^{-1}\circ p} & \ar@/_1.0pc/@{-->}[l]_{s\circ p \circ s_i} f^*\mc O^{n-k}_S}.
% \]
% Letting $e_1,\dots,e_n$ denote the standard basis of $\Gamma(f^*\mc
% O^n_S)$, it's clear that $p$ is determined by the images of
% $e_{n-k+1},\dots,e_n$, and that $p$ has a canonical splitting. If we
% denote by $G_i$ the sheaf obtained by sectionwise forgetting the data
% of the section $s$,  $G_i$ is the
% usual representable subsheaf of the ordinary Grassmannian which sends 
% \[
% G_i(f:X \rightarrow S) = \mapsto Hom(f^*\mc O^k_S,f^*\mc O^{n-k}_S)
% \]
% which is just the space of matrices $\mbb A^{k(n-k)}$. The sheaf $F_i$ is a pullback 
% \[
% \xymatrix{F_i \ar[r]\ar[d] & \Hom(\mc O^{n-k},\mc O^n) \times
%   \Hom(\mc O^k,\mc O^{n-k}) \ar[d] \\ Q_i \ar[r]\ar[d] & \Hom(\mc
%   O^{n-k},\mc O^n) \times \Hom(\mc O^n,\mc O^{n-k}) \ar[d] \\ \{id\}
%   \ar[r] & \Hom(\mc O^{n-k},\mc O^{n-k})}
% \]
% and thus is representable.




% To give intuition for the idea, we
% first define the pseudo-presheaf $\mc F_{(V,\phi)} : \Sm{S} \rightarrow \Set$ by 
% \[
% \mc F_{(V,\phi)}(f: X \rightarrow S) = \{(W,H) \mid W,H \subseteq V_X, \ W \oplus H = V\}.
% \]

% Here $V_x = f^*(V,\phi)$. On morphisms, we would like to define
% \[
% \mc F_{(V,\phi)}(g : X \rightarrow Y)((W,H) \in \mc F_{(V,\phi)}(Y)) =
% (g^*W,g^*H) \in  \mc F_{(V,\phi)}(X),
% \]

% but we run into the standard problem with composition of pullbacks. To make this into an honest presheaf, we recall the strictification
% via a slight modification of Grayson's category of big vector bundles.

% \begin{definition}
% Fix $(V,\phi)$ as above. Let $\mc A$ be a small category equivalent to $(\Sm{S})_{/X}$. A big
% hermitian vector bundle over $X$ is a family 
% \[
% \{(W,H)_Y\}_{(f : Y \rightarrow X) \in \Ob(\mc A)}
% \]
% of pairs as above indexed over $\mc A$ together with a choice of compatible isometries
% \[
% \{\psi_g : g^*(V_Z)\rightarrow V_Y\mid (g : Y \rightarrow Z) \in
% \Mor(\mc A)\}.
% \]

% Note that such a family of isometries induces isometries
% $(g^*W_Z,g^*H_Z) \rightarrow (W_Y,H_Y)$ by restriction.

% Given two big vector bundles $\{(W,H)_Y\}, \{(K,L)_Y\}$ over $X$, a morphism
% between them is given by a family of morphisms $\{h_{Y} : (W,H)_Y
% \rightarrow (K,L)_Y  \}_{(Y \rightarrow X) \in \mc A}$ compatible with the
% chosen pullback isometries.

% Denote the groupoid of big hermitian vector bundles over $X$ by $\BigVB{X}$.
% \end{definition}

% \begin{lemma}
% The assignment $X \mapsto \BigVB{X}$ defines a presheaf of groupoids
% $\BigVB{-}$ on $\Sm{S}$. By applying $\pi_0$, we obtain a presheaf of sets.
% \end{lemma}

% \begin{proof}
% ...
% \end{proof}

% \begin{lemma}
% There's a natural action on $\BigVB{-}$ by sending a big hermitian
% vector bundle $\{(W,H)_Y\}$ on $X$ to the big hermitian vector bundle
% $\{(H^\perp,W^\perp)_Y\}$. 
% \end{lemma}

% \begin{proof}
% We need to check that if $W \oplus H = V$, then $W^\perp \oplus
% H^\perp = V$. There's a map $W^\perp \oplus H^\perp \rightarrow V$
% given by 
% \[
% W^\perp \oplus H^\perp \rightarrow V \oplus V \xrightarrow{+} V
% \]
% where the first map is the sum of the maps in the exact sequences
% defining $W^\perp,H^\perp$. Locally, because $V$ is non-degenerate,
% this map is an isomorphism, thus it's an
% isomorphism. 

% Given a compatible family of
% isometries on $\{(W,H)_Y\}$
% \[
% \{\psi_g : g^*V_Z \rightarrow V_Y \mid (g : Y \rightarrow Z) \in
% \Mor(\mc A)\},
% \]

% we can use the same family of isometries for $\{(H^\perp,W^\perp)_Y\}$
% \todo{Check naturality}.
% \end{proof}

% Now by lemma \ref{lem:action_presheaf}, we get a presheaf of groupoids $\BigVB{-}$
% on $\Sm{S}^{C_2}$. We claim that this presheaf represents
% non-degenerate Hermitian sub-bundles of $(V,\phi)$.

% \begin{lemma}
% Given $X \in \Sm{S}^{C_2}$
% \[
% \Hom(X,\BigVB{-}) \cong \{\text{Groupoid of non-degenerate Hermitian vector bundles
%   over $X$}\}.
% \]
% \end{lemma}

% \begin{proof}
% By construction, 
% \[
% \Hom_{\PreSm}(X,\BigVB{-})
% \]

% is precisely the fixed points of the action $f \mapsto
% a_X(\BigVB{-}(\sigma)(f)$ defined in lemma \ref{lem:action_presheaf}. 
% Such a fixed point is given by
% a big vector bundle $\{(W,H)_Y\}$ such that $\{(W,H)_Y\} =
% \{\sigma_*(H^\perp,W^\perp)\}$, so that $W = \sigma_*(H^\perp)$. Now
% if we forget the module structure, then $\sigma_*(H^\perp) = H^\perp$,
% so that, again forgetting the module structure, $W^\perp = H$. It
% follows that $W$ is complemented, and hence non-degenerate as a
% sub-bundle of $V_X$. 

% \end{proof}

\subsection{The \'Etale Classifying Space}

Fix a scheme $S$ with trivial involution and 2 invertible, and let $(V,\phi)$ be a
(possibly degenerate) hermitian vector bundle over $S$. For a
$C_2$-scheme $f : X \rightarrow S$, let
\[
\mc S(V,\phi)(X)
\]
be the category of non-degenerate hermitian sub-bundles of $f^*V$. A
morphism in this category from $E_0$ to $E_1$ is an isometry not
necessarily compatible with the embeddings $E_0,E_1 \subseteq
V$. Using pullbacks of quasi-coherent modules, we turn $\mc S$ into a presheaf of categories on
$\Sm{S}^{C_2}$. For integer $d \geq 0$, define
\[
\mc S_d(V,\phi)  \subset \mc S(V,\phi)
\]
to be the presheaf which on a $C_2$-scheme $f : X \rightarrow S$
assigns the full subcategory of non-degenerate hermitian sub-bundles
of $(f^*V,f^*\phi)$ which have constant rank $d$. The associated
presheaf of objects is $\RGr_d(V,\phi)$.

Note that the object $V = (V,0) \in \mc S_{|V|}(V \perp H^\infty)$ has
automorphism group $O(V)$. Thus we get an inclusion $O(V,\phi)
\rightarrow \mc S_{|V|}(V \perp H^\infty) $, where $O(V)$ is the
isometry group considered as a category on one object. After isovariant \'etale
sheafification, this inclusion becomes an equivalence; this follows
from our remarks above that on the points in the isovariant \'etale
topology, hermitian vector bundles are isomorphic to $\mbb H^n(P)$ for
some $n$, where $P$ is a hermitian vector bundle over a strictly
henselian local ring. Since all hermitian vector bundles over strictly
henselian local rings (with 2 invertible) are trivial, hermitian vector bundles over the
points in the isovariant \'etale topology are completely determined by
rank. 

Upon applying the nerve, we get maps of simplicial presheaves $BO(V)
\rightarrow B\mc S_{|V|}(V\perp \mbb H^\infty)$ which is a weak
equivalence in the isovariant \'etale topology. Abusing notation, let $B_{isoEt}O(V)$
denote a global fibrant replacement of $B\mc S_{|V|}(V\perp \mbb H^\infty)$ in the isovariant
\'etale topology so that we get a sequence of weak equivalences
\[
BO(V) \rightarrow B\mc S_{|V|}(V\perp \mbb H^\infty) \rightarrow B_{isoEt}O(V).
\]

\begin{lemma} \label{weqaff}
Let $(V,\phi)$ be a non-degenerate hermitian vector bundle over a
scheme $S$ with trivial involution and $\frac{1}{2} \in \Gamma(S,\mc
O_S)$. Then for any affine $C_2$-scheme over $S$, $\Spec R$, the map
\[
B\mc S_{|V|}(V\perp \mbb H^\infty)(R) \rightarrow B_{isoEt}O(V)(R)
\]
is a weak equivalence of simplicial sets. In particular, the map
\[
B\mc S_{|V|}(V\perp \mbb H^\infty) \rightarrow B_{isoEt}O(V)
\]
is a weak equivalence in the equivariant Nisnevich topology, and hence an
equivalence after $C_2$ motivic localization. 
\end{lemma}

\begin{proof}
% Counterexample: Let $S = \Z[1/2]$ and $R = \mbb R$. Then $O(\mbb H_S)$
% is represented by the $C_2$-scheme $GL_{2}$ with inverse adjoint
% action. It follows that $O(\mbb H_S)$ is an isovariant \'etale sheaf of groups,
% and hence $BO(\mbb H_s)$ is fibrant in the isovariant \'etale
% topology. If the map
% \[
% B\mc S_{|V|}(V\perp \mbb H^\infty)(R) \rightarrow B_{isoEt}O(V)(R)
% \]
% is an equivalence of simplicial sets, then because the composite
% \[
% BO(V) \rightarrow B\mc S_{|V|}(V\perp \mbb H^\infty) \rightarrow B_{isoEt}O(V)
% \]
% is a weak equivalence of fibrant objects, it's a global
% equivalence. It would follow by 2-3 that the map 
% \[
% BO(V)(R) \rightarrow B\mc S_{|V|}(V\perp \mbb H^\infty)(R)
% \]
% is a weak equivalence of simplicial sets. This is false as it fails to
% be surjective on $\pi_0$.




Each Hermitian vector bundle $W  \in S_{|V|}(V\perp \mbb H^\infty)(R)$
gives rise to an $ O(V)$-torsor via $W \mapsto Isom(V,W)$. Note
that this is an $O(V)$-torsor because etale locally, $W \cong V$,
so that etale locally $Isom(V,W) \cong Isom(V,V) \cong 
O(V)$. Because hermitian vectors bundles are isovariant etale locally
determined by rank, the same proof as the vector bundle case shows
that the category of $\mc O(V)$ torsors is equivalent to the category
of Hermitian vector bundles. Because over an affine scheme, every
hermitian vector bundle is a summand of a hyperbolic module, it
follows that $S_{|V|}(V\perp \mbb H^\infty)(R)$ is equivalent to the
category of (etale) $\mc O(V)$ torsors. 

Let $\mc F : \Sm{S}^{C_2} \rightarrow Gpd$ be the sheaf which assigns
to $f : X \rightarrow S$ the groupoid of $ O(f^*V)$-torsors. The
construction $W \mapsto Isom(f^*V,W)$ described above defines a
functor $S_{|V|}(V\perp \mbb H^\infty) \rightarrow \mc F$ which is an
equivalence when evaluated at affine $C_2$-schemes. It follows that
there's a sequence
\[
B S_{|V|}(V\perp \mbb H^\infty) \rightarrow B\mc F \rightarrow B_{isoEt}O(V)
\]

where the first map is a weak equivalence of simplicial sets when
evaluated at affine $C_2$-schemes, and by \cite{Jar01} theorem 6, the second map is a
weak equivalence of simplicial sets when evaluated at any
$C_2$-scheme. 
\end{proof}


\begin{definition}
Following \cite{SchTri}, let
\[
\mc S_\bullet = \colim_{V \subset \mbb H^\infty_S} \mc S_{|V|}(V \perp
\mbb H^\infty)
\]
where similarly to the definition of $\RGr$, for $V \subset V'$ the
functor 
\[
\mc S_{|V|}(V \perp
\mbb H^\infty) \rightarrow \mc S_{|V'|}(V' \perp
\mbb H^\infty)
\]
is defined on objects by $E \mapsto E \perp V'-V$ and on morphisms by
$f \mapsto f \perp 1_{V'-V}$. 
\end{definition}

\begin{definition}
Define
\[
O = \colim_{W \subseteq \mbb H^\infty_S} O(W).
\]
Because the nerve construction commutes with filtered colimits, and
because filtered colimits of globally fibrant objects are globally
fibrant (follows from the fact that filtered colimits of Kan complexes
are Kan complexes), define by abuse of notation
\[
B_{isoEt}O = \colim_{W \subseteq \mbb H^\infty_S} B_{isoEt}O(W).
\]
\end{definition}
% Inclusion of discrete categories defines a functor
% \[
% \RGr_{|V|}(V \perp \mbb H^\infty) \rightarrow \mc S_{|V|}(V \perp \mbb H^\infty)
% \]
% and after applying the nerve we get a sequence
% of maps
% \[
% B\RGr_{|V|}(V \perp \mbb H^\infty) \rightarrow B\mc S_{|V|}(V \perp
% \mbb H^\infty) \rightarrow B_{isoEt}O(V)
% \]
% where the second map is a weak equivalence on affine schemes by Lemma
% \ref{weqaff}. Let
% \[
% O = \colim_{V \subset \mbb H^\infty} O(V)
% \]
% and note that this colimit can be computed either in the category of
% groups or the category of small categories (where $O(V)$ is embedded
% as a one object category). Since the nerve construction commutes with
% filtered colimits, we
% have
% \[
% BO(V) \cong \colim_{V \subset \mbb H^\infty} BO(V)
% \]
% and 

\begin{theorem}
Let $R$ be a regular noetherian ring with involution which is either connected or
hyperbolic. Then there's an equivalence of simplicial sets
\[
B\RGr_\bullet(\Delta R) \rightarrow |B \mc S_\bullet (\Delta R)|
\]
where $\Delta R$ denotes the simplicial ring with involution $[n] \mapsto
R[x_0,\dots,x_n]/(\sum x_i -1)$. 
\end{theorem}


\begin{lemma}
Let $\Iso_d(R)$ denote the set of isometry classes of finitely-generated, non-degenerate
hermitian vector bundles over $R$. The map
\[
\colim_{V \subset \mbb H^\infty_R} \Iso_{|V|}(R) = \coprod_{V \subset
  \mbb H^\infty} \Iso_{|V|}/\sim \cong \widetilde{GW}_{[0]}(R)
\]
sending $(V,W) \in \Iso_{|V|}$ to $[V] - [W]$ is an isomorphism. 
Here $\widetilde{GW}_{[0]}(R)$ is the kernel of the rank map $GW_0(R)
\rightarrow \Z$.
\end{lemma}

\begin{proof}
First, note that the map is well-defined. If there's an inclusion $V
\hookrightarrow T \hookrightarrow \mbb H^\infty_R$, then 
\[
(T,W \perp (T-V)) \mapsto [T] - [W + T - V] = [V] - [W].
\]
Furthermore, by definition if $W \in \Iso_{|V|}(R)$, then $\rk(V) =
\rk(W)$ and hence $[V]-[W] \in \ker(\rk) : GW_0(R) \rightarrow \Z$. 

If $[V] - [W] = 0$ in $GW_0$, then there's a non-degenerate bundle
$[K]$ such that $V \perp K \cong W \perp K$. It follows that $(V,W) \in
\Iso_{|V|}(R) \sim (V \perp K, W \perp K) = (V \perp K, V \perp K)
\sim (0,0\in \Iso_{|0|}(R))$ so that the map is
injective. Surjectivity is clear because over a ring, every bundle is,
up to isometry, a sub-bundle of $\mbb H^\infty_R$.
\end{proof}

Now, note that there are maps of sets
\[
\RGr_d(V \perp \mbb H^\infty_R) \rightarrow \Iso_d(R): E \mapsto [E]
\]
and (considering a set as a discrete category) maps of categories
\[
\mc S_d(V \perp \mbb H^\infty_R) \rightarrow \Iso_d(R) : E \mapsto [E].
\]

These maps fit into cartesian squares

\[
\xymatrix{\RGr_V(V \perp \mbb H^\infty) \ar[r]\ar[d] & \RGr_{|V|}(V
  \perp \mbb H^\infty) \ar[d] \\ \star \ar[r]^V & \Iso_{|V|}(R)}
\]
and 
\[
\xymatrix{\mc S_V(V \perp \mbb H^\infty) \ar[r]\ar[d] & \mc S_{|V|}(V
  \perp \mbb H^\infty) \ar[d] \\ \star \ar[r]^V & \Iso_{|V|}(R)}
\]

where $\RGr_V(V \perp \mbb H^\infty_R)$ is the subset of $\RGr_{|V|}(V
\perp \mbb H^\infty_R)$ of bundles isometric to $V$, and similarly
$\mc S_V(V \perp \mbb H^\infty) \subseteq \mc S_{|V|}(V \perp \mbb H^\infty)$ is the full subcategory whose objects
correspond to the set $\RGr_V(V \perp \mbb H^\infty_R)$.

Taking colimits over non-degenerate subspaces $V \subset \mbb
H^\infty_R$ and using the standard facts that the nerve functor commutes with filtered colimits and
that filtered colimits of cartesian diagrams are cartesian, we get cartesian diagrams of simplicial sets
\[
\xymatrix{B\RGr_{[0]}(V \perp \mbb H^\infty)(R) \ar[r]\ar[d] & B\RGr_\bullet(V
  \perp \mbb H^\infty)(R) \ar[d] \\ B\star \ar[r]^V & B\widetilde{GW}_{[0]}(R)}
\]
and 
\[
\xymatrix{B\mc S_{[0]}(V \perp \mbb H^\infty)(R) \ar[r]\ar[d] & B\mc S_\bullet(V
  \perp \mbb H^\infty)(R) \ar[d] \\ B\star \ar[r]^V & B\widetilde{GW}_{[0]}(R)}
\]

where the upper left corners are just defined as the respective colimits. 

\begin{lemma}
The diagrams
\[
\xymatrix{B\RGr_{[0]}(V \perp \mbb H^\infty)(\Delta R) \ar[r]\ar[d] & B\RGr_\bullet(V
  \perp \mbb H^\infty)(\Delta R) \ar[d] \\ B\star \ar[r]^V & |B\widetilde{GW}_{[0]}(\Delta R)|}
\]
and 
\[
\xymatrix{|B\mc S_{[0]}(V \perp \mbb H^\infty)(\Delta R)| \ar[r]\ar[d] & |B\mc S_\bullet(V
  \perp \mbb H^\infty)(\Delta R)| \ar[d] \\ B\star \ar[r]^V &
  |B\widetilde{GW}_{[0]}(\Delta R)|}
\]
are homotopy cartesian over any regular ring $R$ with involution such that
non-degenerate hermitian vector bundles have constant rank.
\end{lemma}

\begin{proof}

% We claim that $\colim_{V \subset \mbb H^\infty_R} \mathrm{Iso}_{|V|}(R)
% \cong \widetilde{GW}_0(R)$, the zeroth reduced Grothendieck-Witt group
% of $R$. Indeed, our assumptions imply that $\widetilde{GW}_0(R)$ is
% the kernel of the rank map $GW_0(R) \rightarrow \Z$, and there's a map
% \[
% \colim_{V \subset \mbb H^\infty_R} \mathrm{Iso}_{|V|}(R) \rightarrow
% \widetilde{GW}_0(R) \qquad (E,V) \mapsto [E] - [V].
% \]

% One can quickly check that the map is well-defined and an isomorphism
% (here it's crucial that we only talk about Grothendieck-Witt groups of
% rings so that all non-degenerate bundles embed up to isometry in
% hyperbolic space). 

First, note that before applying the diagonal functor $| - |$, these
diagrams are cartesian diagrams of bisimplicial sets. This follows
simply because limits are computed object-wise in functor
categories. For the same reason, we get a cartesian diagram after
applying $| - |$. Thus to prove that the diagrams are homotopy
cartesian (in the standard model structure on simplicial sets), it
suffices to prove that the bottom horizontal map is a fibration. If $R$
is a regular ring with involution, then $GW_0(R[t]) \cong GW_0(R)$
where the involution on $R[t]$ is on the coefficients of a
polynomial. It follows that the reduced Grothendieck-Witt group is
also homotopy invariant. It follows that the simplicial set in the bottom right
corner of both diagrams is discrete. A map of discrete simplicial sets
is a Kan fibration, since a map from a simplicial set to a discrete
simplicial set is completely determined by the map on zero simplices,
and the zero simplices of an $n$-horn and $n$-simplex for $n \geq 1$ agree.
\end{proof}



Via inclusion of zero simplices, there is a map of homotopy fibrations
\begin{equation}\label{dia:Grass_GW}
\begin{gathered}
\xymatrix{B\RGr_{[0]}(V \perp \mbb H^\infty)(\Delta R) \ar[r]\ar[d] & B\RGr_\bullet(V
  \perp \mbb H^\infty)(\Delta R) \ar[d] \ar[r] &
  |B\widetilde{GW}_{[0]}(\Delta R)| \ar[d]^{id} \\
|B\mc S_{[0]}(V \perp \mbb H^\infty)(\Delta R)| \ar[r]& |B\mc S_\bullet(V
  \perp \mbb H^\infty)(\Delta R)| \ar[r] &
  |B\widetilde{GW}_{[0]}(\Delta R)|}. 
\end{gathered}
\end{equation}

In order to conclude that the map $B\RGr_\bullet(\Delta R) \rightarrow
|B \mc S_\bullet (\Delta R)|$ is a weak equivalence of simplicial
sets, it suffices to check two things:
\begin{itemize}
\item in diagram \ref{dia:Grass_GW}, the map on fibers $B\RGr_{[0]}(V
  \perp \mbb H^\infty) \rightarrow |B\mc S_{[0]}(V \perp \mbb H^\infty)(\Delta R)| $is a weak
  equivalence (the map on bases is the identity),
\item in diagram \ref{dia:Grass_GW}, the maps are maps of
  $E_\infty$-spaces.
\end{itemize}

\begin{remark}
Note that even over rings where $|B\widetilde{GW}_{[0]}(V\perp \mbb H^\infty)(\Delta R)|$ is
a constant simplicial set, $B\RGr_{[0]}(V \perp \mbb H^\infty)$ will
not be. This is simply because there are more hermitian vector bundles over
$R[x]$ than over $R$ when we don't mod out by isometry. 

\end{remark}

\begin{example}
For an
explicit example that demonstrates why $|B\RGr_{[0]}(\Delta
R)|$ has a hope of being connected (if it was discrete it would in
general not be), let $R = \mbb R$ with trivial involution, and
consider the simplicial set
\[
|B\RGr_{[0]}(\langle 1 \rangle_{\mbb R} \perp \mbb H_{\mbb R}^\infty)(\Delta R)|.
\]

Consider the two split surjections
\[
X = \langle 1 \rangle_{\mbb R} \perp \mbb H_{\mbb R}^\infty
\xrightarrow{\pi_{\langle 1 \rangle}} \langle 1 \rangle
\]
and
\[
Y = \langle 1 \rangle_{\mbb R} \perp \mbb H_{\mbb R}^\infty
\xrightarrow{\pi_2 \oplus \pi_3} \mbb H \xrightarrow{+} \mbb R
\]
where the second surjection is split by $\frac{1}{2}\Delta$. 
Now consider the split surjection over $R[x]$ given by
\[
T = \langle 1 \rangle_{\mbb R} \perp \mbb H_{\mbb R}^\infty
\xrightarrow{\pi_{\langle 1 \rangle} \oplus \pi_2 \oplus \pi_3}
\langle 1 \rangle \oplus \mbb H \xrightarrow{+} R[x]
\]
where the last surjection is split by the map sending $1 \mapsto
(x,\frac{1}{2}(1-x),\frac{1}{2}(1-x))$. We claim
that under the two maps $R[x] \rightarrow R$, $x \mapsto 0, \xmapsto
1$, the split surjection $T$ restricts to $X$ and $Y$. Indeed, this is
just the fact that given an $R[x]$-module structure on $R$ via the map
$\eta_t: R[x] \rightarrow R$, $x \mapsto t$, as well as a map $R[x]
\rightarrow R[x]$. $1 \mapsto x$, 
the induced map $R \cong R[x]\otimes_{R[x]} R \xrightarrow{x \otimes id} R[x] \otimes_{R[x]}
R \cong R$ is multiplication by $\eta_t(x)$.
\end{example}

We proceed to prove that the map on fibers is a weak equivalence by
presenting the domain and codomain as free quotients of contractible
spaces. To set up the relevant group actions, we need the following lemma.


\begin{lemma}\label{lem:incHE}
Let $V$ be a nondegenerate hermitian vector bundle over a commutive
ring with involution $(R,\sigma)$ such that $\frac{1}{2} \in R$. Then
the inclusion $\mbb H^\infty \subset V \perp \mbb H^\infty$ induces a
homotopy equivalence of simplicial groups
\[
O(\mbb H^\infty_{\Delta R}) \rightarrow O(V \perp \mbb
H^\infty_{\Delta R}) \qquad A \mapsto 1_V \perp A.
\]
\end{lemma}

\begin{proof}
First, assume that $V = \mbb H$. Consider the map $j : O(\mbb H^n)
\rightarrow O(\mbb H^{2n+2})$ sending $A$ to $1_H \perp A \perp
1_{\mbb H^{n+1}}$. We claim that this is naively $\mbb A^1$ homotopic
to the inclusion $i : O(\mbb H^n) \rightarrow O(\mbb H^{2n+2})$, $i(A)
= A \perp 1_{\mbb H^{n+2}}$ which defines the colimit $O(\mbb
H^\infty)$. Let $g = \begin{pmatrix}
0 & I_{2n} & 0 \\
I_2 & 0 & 0 \\
0 & 0 & I_{2n+2}
\end{pmatrix}$ where $I_n$ denotes an $n \times n$ identity
matrix. Then $i = gjg^{-1} = gjg^t$. Because $g$ corresponds to an
even permutation matrix, it can be written as a product of elementary
matrices, each of which is naively $\mbb A^1$ homotopic to the
identity. It follow that $g$ is naively $\mbb A^1$ homotopic to the
identity, and hence the induced maps $i,j : O(\mbb H^n_{\Delta R})
\rightarrow O(\mbb H^{2n+2}_{\Delta R})$ are simplicially homotopic
via a base-point preserving homotopy. It follows that $i,j$ induce the
same map on homotopy groups, so that $j_* = i_*: \pi_kO(\mbb
H^\infty_{\Delta R}) = \colim_n \pi_k O(\mbb H^n_{\Delta R})
\rightarrow \pi_k O(\mbb H^\infty_{\Delta R})$ is the colimit of a map
corresponding to a cofinal inclusion of diagrams, and hence is an
isomorphism on all simplicial homotopy groups. Because simplicial
groups are Kan complexes, it follows that $j$ is a homotopy
equivalence, and the claim is proved when $V = \mbb H$.

Now a trivial induction shows that the lemma holds when $V = \mbb
H^n$. In general, choose an embedding $V \subseteq \mbb H^n$, and
consider the sequence of maps
\[
O(\mbb H^\infty_{\Delta R}) \rightarrow O(V \perp \mbb
H^\infty_{\Delta R}) \rightarrow O(\mbb H^n \perp \mbb H_{\Delta R}^\infty)
\rightarrow O(\mbb H^n \perp V \perp \mbb H^\infty_{\Delta R}).
\]

The composites $O(\mbb H^\infty_{\Delta R}) \rightarrow O(\mbb H^n
\perp \mbb H^\infty)$ and $O(V \perp \mbb
H^\infty_{\Delta R}) \rightarrow O(\mbb H^n \perp V \perp \mbb
H^\infty_{\Delta R})$ are weak equivalences, so by 2 out of 6 the
first map is a weak equivalence. Because it is a map of simplicial
groups it is a homotopy equivalence. 
 
\end{proof}

For nondegenerate hermitian vector bundles $(V,\phi_V), (W,\phi_W)$ and a commutative
$R$-algebra with involution $(A,\sigma)$, let
\[
\St(V,W)(A)
\]

be the set of $A$-linear isometric embeddings $f : V_A \rightarrow
W_A$. Given a map $A \rightarrow B$ of commutative $R$-algebras with involution, tensoring over $R$ with $B$ makes
$\St(V,W)(-)$ a presheaf on commutative $R$-algebras with involution. There's a
transitive left action of $O(V \perp \mbb H^\infty)$ on $\St(V,V \perp \mbb
H^\infty)$ given by $(f,g) \mapsto f \circ g$. Let $i_V$ denote the
isometric embedding $V \hookrightarrow V \perp \mbb H^\infty : v
\mapsto (v,0)$. The stabilizer of $i_V$ is the subgroup $O(\mbb
H^\infty) \subset O(V \perp \mbb H^\infty)$ where the inclusion map is
$A \mapsto 1_V \perp A$.

It follows that there's an isomorphism of presheaves of sets 
\[
O(\mbb H^\infty)\backslash O(V \perp \mbb  H^\infty) \cong
\St(V,V\perp \mbb H^\infty) \qquad f \mapsto f \circ i_V.
\]

Now Lemma \ref{lem:incHE} shows that the map $O(\mbb H^\infty_{\Delta
  R}) \rightarrow O(V \perp \mbb H^\infty_{\Delta R})$ is an
equivariant map which is a non-equivariant homotopy equivalence. The
simplicial group $O(\mbb H^\infty_{\Delta R})$ acts freely on both the
domain and codomain, so that the quotients $O(\mbb H^\infty_{\Delta
  R})\backslash O(V \perp \mbb H^\infty_{\Delta R})$ and $O(\mbb H^\infty_{\Delta
  R})\backslash O(\mbb H^\infty_{\Delta R})$ are homotopy equivalent.

Together with the isomorphism of simplicial sets
\[
O(\mbb H^\infty_{\Delta
  R})\backslash O(V \perp \mbb H^\infty_{\Delta R}) \cong \St(V,V
\perp \mbb H^\infty_{\Delta R})
\]

it follows that $\St(V,V\perp \mbb H^\infty_{\Delta R})$ is a
contractible for a commutative ring $(R,\sigma)$ with involution and
$\frac{1}{2} \in R$. Morever, this simplicial set is fibrant because
$G/H$ is fibrant for a simplicial group $G$ and subgroup $H$.

Now we move to identifying $\RGr_d(V)$ as a quotient of a contractible
space by a free group aciton. Let $V$ be a non-degenerate hermitian
vector bundle over a ring $R$ with involution. Then the group $O(V)$
acts on the right on $\St(V,U)$ by precomposition. The map $\St(V,U)
\rightarrow \RGr_V(U) : f \mapsto \im(f)$ factors through the quotient
$\St(V,U)/O(V)$. The map is clearly surjective, and hence furnishes an
isomorphism of sets
\[
\St(V,U)/O(V) \cong \RGr_V(U) \qquad f \mapsto \im(f).
\]

In particular, there's an isomorphism of presheaves of sets $\St(V,V \perp \mbb
H^\infty)/O(\mbb H^\infty) \cong \RGr_V(U)$.

Now, for a non-degenerate hermitian vector bundle $V$ over a ring with
involution $R$, and let $U$ be a possible degenerate hermitian form
over $R$. Define $\mc E_V(U)$ to be the category whose objects are
$R$-linear maps $V \rightarrow U$ of hermitian forms, and whose
morphisms from two objects $a : V \rightarrow U$ and $b : V
\rightarrow U$ are maps $c : \im(a) \rightarrow \im(b)$ making the
diagram
\[
\xymatrix{V \ar[r]^a\ar[dr]_b & \im(a) \ar[d]^c\\ & \im(b)}
\]

commute.

There's a natural right action of $O(V)$ on $\mc E_V(U)$ which on
objects sends
\[
\mc E_V(U) \times O(V) \rightarrow \mc E_V(U) : (a,g) \mapsto ag
\]
and which morphisms is the trivial action.

Then clearly there's an isomorphism

\[
\mc E_V(U)/O(V) \cong \mc S_V(U) \qquad a \mapsto \im(a).
\] 

\begin{lemma}
The category $\mc E_V(V \perp \mbb H^\infty)$ is contractible.
\end{lemma}

\begin{proof}
The category is nonempty and every object is initial.
\end{proof}

Now we show that the map on fibers in \ref{dia:Grass_GW} is a weak
equivalence. The map of simplicial sets
\[
\St(V,V\perp \mbb H^\infty)(\Delta R) \rightarrow \mc E_V(V \perp \mbb
H^\infty)(\Delta R)
\]

is $O(V_{\Delta R})$ equivariant and a weak equivalence after
forgetting the action. Furthermore, $O(V_{\Delta R})$ acts freely on
both sides, so that the induced map on quotients $\RGr_V(V \perp \mbb
H^\infty_{\Delta R}) \rightarrow \mc S_V(V \perp \mbb H^\infty_{\Delta
  R})$ is also a weak equivalence. 

As an aside, the inclusion $BO(V) \subset B\mc S_V(V \perp \mbb
H^\infty)$ is a weak equivalence since $\mc S_V(V \perp \mbb
H^\infty)$ is a connected groupoid. 

\subsubsection{Showing that digram \ref{dia:Grass_GW} is a diagram in
  $E_\infty$-spaces}

For a commutative ring with involution $(R,\sigma)$, let $\mathscr
E(n)(R)$ be the set
\[
\mathscr E(n)(R) = \lim_{V \subset \mbb H^\infty_R} \St(V^{\perp n} , \mbb H^\infty_R).
\]

where limit is over non-degenerate subspaces of $\mbb H^\infty$. The
permutation group $\Sigma_n$ acts by permuting the component
subspaces. The maps in the limit are equivariant with respect to this
free action, and hence there's an induced free action on the
limit. Now if $V \subseteq W$, then the map $\St(V^n,\mbb
H^\infty_{\Delta R}) \rightarrow \St(W^n,\mbb H^\infty_{\Delta R})$ is
a Kan fibration, and hence
\[
\mathscr E(n)(\Delta R) = \lim_k \St(\mbb H^k \perp \cdots \perp \mbb
H^k,\mbb H^\infty)(\Delta R)
\]

is a tower of Kan fibrations with each object fibrant. It follows that
this limit is a homotopy limit, and the Milnor sequence implies that
$\mathscr E(n)(\Delta R)$ is contractible. The same reasoning as above
implies that the action of $\Sigma_n$ is free. 

Now, define the structure maps of the operad by

\[
\mathscr E(k) \times \mathscr E(j_1) \times \cdots \times \mathscr
E(j_k) \rightarrow \mathscr E(j_1 + \cdots + j_k) : f,g_1,\cdots,g_k
\mapsto f \circ (g_1 \perp \cdots \perp g_k).
\]

It follows that $\mathscr E(\Delta R)$ is an $E_\infty$-operad in the
category of simplicial sets.

\begin{proposition}
For any commutive ring with involution $(R,\sigma)$ such that
$\frac{1}{2} \in R$, the map given by inclusion of 0-simplices
\[
\RGr_\bullet(\Delta R) \rightarrow \mc S_\bullet(\Delta R)
\]
is a map of group complete $E_\infty$-spaces.
\end{proposition}

\begin{proof}
Write
\[
\mc S_\bullet = \colim_{V \subset \mbb H^\infty} \mc S_{|V|}(V^- \perp V^+)
\]
where $V^-$ and $V^+$ are two copies of $V$ and for $V \subset W$ the
transition map is defined by
\[
\mc S_{|V|}(V^- \perp V^+) \rightarrow \mc S_{|W|}(W^- \perp W^+) : E
\mapsto (W-V)^- \perp E, g \mapsto 1_{(W-V)^-} \perp g.
\]

Now, the action of $\mathscr E$ on $\mc S_\bullet$ is defined by
\[
\St(V_1 \perp \cdots \perp V_k,W) \times \mc S_{|V_1|}(V_1^- \perp
V_1^+) \times \cdots \times \mc S_{|V_k|}(V_k^- \perp V_k^+)
\rightarrow \mc S_{|W|}(W^- \perp W^+)
\]

where for $g \in \St(V_1 \perp \cdots \perp V_k,W)$, the functor
\[
\mc S_{|V_1|}(V_1^- \perp V_1^+) \times \cdots \times \mc
S_{|V_k|}(V_k^- \perp V_k^+) \rightarrow \mc S_{|W|}(W^- \perp W^+)
\]
sends the object $(E_1,\dots,E_k)$ to 
\[
(W - g(V_1\perp \cdots \perp V_k))^- \perp g(E_1 \perp \cdots \perp E_k)
\]
and the map $(e_1,\dots,e_k): (E_1,\dots,E_k) \rightarrow
(E_1',\dots,E_k')$ to
\[
1_{(W - g(V_1 \perp \cdots \perp V_k))^-} \perp g|_{E_1'} \circ e_1
\circ g^{-1}|_{E_1} \perp \cdots \perp g|_{E'_k} \circ e_k \circ g^{-1}|_{E_k}.
\]

To see that the spaces are group complete, note that the homotopy
fiber sequences above imply that the $\pi_0$ of both spaces is
$\widetilde GW_0(R)$. Indeed, it's straightforward to check that
$\pi_0(\RGr_{[0]}(\Delta R),x) = \{\ast\}$ for any choice of basepoint
$x$, which implies that the right
maps in \ref{dia:Grass_GW} are an injection on $\pi_0$. The maps on
zero simplices are clearly surjective, and hence the maps on $\pi_0$
must be surjective. 
\end{proof}

Now that we've checked that the maps in \ref{dia:Grass_GW} are maps of
$E_\infty$-spaces and that they're weak equivalences on base and
fiber, we can conclude that the map on total spaces is an equivalence.

\begin{corollary}
Let $(R,\sigma)$ be a regular ring with involution such that
non-degenerate hermitian vector bundles have constant rank, and such
that $\frac{1}{2} \in R$. Then the map
\[
\RGr_\bullet(\Delta R) \rightarrow \mc S_\bullet(\Delta R)
\]
is a weak equivalence of simplicial sets.
\end{corollary}



\subsection{The Grothendieck-Witt space}


For a ring $R$ with involution $\sigma$, there's an associated
category $S(R)$ with duality given by vector bundles
with their canonical duality and vector bundle morphisms as
morphisms. The subcategory of hermitian objects and isometries is symmetric monoidal under $\perp$, and the translations
$A \mapsto A \perp B$ are faithful. Quillen's $S^{-1}S(R)$ construction
yields a symmetric monoidal category with objects pairs $(A_0,A_1)$ in
$S$ and morphisms $(A_0,A_1) \rightarrow (B_0,B_1)$ equivalence
classes $[C,a_0,a_1]$ with $a_i : C \perp A_i \rightarrow B_i$ an
isometry. Two morphisms $[C,a_0,a_1], [C',a_0',a_1']$ are equivalent
if there exists an isometry $f : C \cong C'$ such that $a_i'\circ
(1_{A_i} \perp f) = a_i$. Unfortunately, this category is neither
small (in general) nor strictly functorial in
the underyling $C_2$-scheme. 

\begin{definition}
Let $(R,\sigma)$ be a ring with involution, and let
\[
\mathscr GW(R,\sigma) \subset S^{-1}S(R,\sigma)
\]
be the full subcategory whose objects are pairs $(A,B)$ where $A
\subset \mbb H^\infty_R \perp \mbb H^\infty_R$ and $B \subset (\mbb
H^\infty_R)^{\perp 3}$ are finitely genereated nondegenerate
subspaces. 

Let $(X,\sigma)$ be a $C_2$-scheme, and let
\[
\mathscr GW(X,\sigma) = \mathscr GW(\Spec \Gamma(X),\sigma).
\]
\end{definition}

Note that $\mathscr GW(R,\sigma) \hookrightarrow S^{-1}S(R,\sigma)$ is
an equivalence because over a ring, every non-degenerate vector bundle
is a summand of hyperbolic space. Thus by \cite{Schder}, Theorem A.1, there's an equivalence
$\mathscr GW(R,\sigma) \cong \Omega^\infty GW(R,\sigma)$ for any ring
with involution $(R,\sigma)$. 

\subsection*{Homotopy colimits of categories}

\begin{definition}
Let $\mc C$ be a small category, and let $J : \mc C \rightarrow \Cat$
a functor into the category of small categories. The homotopy colimit
\[
\hocolim_{\mc C} J
\]
is the category whose objects are pairs $(X,A)$ with $X$ an object of
$\mc C$ and $A$ an object of $J(X)$. A map from $(X,A)$ to $(Y,B)$ is
a pair $(x,a)$ where $x : X \rightarrow Y$ is a map in $\mc C$ and $a
: J(x)(A) \rightarrow B$ is a map in $J(Y)$. Composition $(y,b) \circ
(x,a)$ is the map $(y \circ x, b \circ J(y) \circ a)$. 
\end{definition}

We recall some notation from \cite{Gra76}.

\begin{definition}
Let $S$ be a symmetric monoidal category acting on another category
$X$. The category $\langle S, X\rangle$ is by definition the category
whose objects are the objects of $X$, and whose morphisms $F
\rightarrow G$ are
isomorphism classes of tuples $(F,G,A,A+F \rightarrow G)$ with $A \in
S$ and $F,G$ in $X$. An isomorphism of tuples is an isomorphism $A
\cong A'$ which makes the diagram
\[
\xymatrix{A+F \ar[rr]^{\sim} \ar[dr]&& A' + F \ar[dl] \\ &G&}
\]
commute.
\end{definition}

Now consider the category $\mc S(\mbb H_R^\infty)$ of finitely generated
non-degenerate subspaces of $\mbb H_R^\infty$. It's symmetric monoidal
via $\perp$, and thus it acts on itself by translation. Then $\langle
\mc S(\mbb H_R^\infty), \mc S(\mbb H_R^\infty) \rangle$ is the category
whose objects are finitely generated non-degenerate subspaces of $\mbb
H_R^\infty$, and whose morphisms $W \rightarrow T$ are isomorphism classes of isometries
$V \perp W \rightarrow T$. 

We claim that the morphisms correspond to
isometric embeddings $W \hookrightarrow T$ which don't necessarily
commute with the embeddings into  $\mbb H^\infty$. First, given an isometry
$\phi : V \perp W \rightarrow T$,  $\phi|_W : W \rightarrow T$ is an
isometric embedding. Given two isomorphic morphisms $W \rightarrow T$
(as defined above), they necessarily
restrict to the same map on $W$ so that there's a well-defined map of
sets from the morphisms in $\langle
\mc S(\mbb H_R^\infty, \mc S(\mbb H_R^\infty)\rangle$ to isometric
embeddings. Given an isometric embedding $\phi : W \hookrightarrow T$,
because $W$ is non-degenerate there's a decomposition $T = \phi(W)
\perp (\phi(W))^\perp$. It follows that there's an isometry 
$(\phi(W))^\perp \perp W \rightarrow T$, yielding a morphism in $\langle
\mc S(\mbb H_R^\infty), \mc S(\mbb H_R^\infty) \rangle$.

\begin{definition}
Define a functor $\mc I : \langle
\mc S(\mbb H_R^\infty), \mc S(\mbb H_R^\infty) \rangle \rightarrow
\Cat$ which on objects is defined by $\mc I(V) = \mc S_{|V|}(V \perp
\mbb H_R^\infty)$ and given a morphism $g : V
\hookrightarrow W$, $\mc I(g)$ is the functor
\[
\mc I(g) : \mc S_{|V|}(V \perp
\mbb H_R^\infty) \rightarrow \mc S_{|W|}(W \perp
\mbb H_R^\infty),
\]
\begin{align*}
&E \mapsto (W - g(V)) \perp (g\perp id)(E)\\
&e \mapsto id_{W-g(V)} \perp geg^{-1}.
\end{align*}

Now, let 
\[
\widetilde{\mathscr GW}(R) = \hocolim \mc I.
\]

To spell this out, the objects of $\widetilde{\mathscr GW}(R)$ are pairs $(V,W)$ with $V
\subseteq \mbb H^\infty_R$ a finitely generated non-degenerate
subspace and $W \subset V \perp \mbb H^\infty_R$ a finitely generated
non-degenerate subspace of constant rank $|V|$.

A morphism $(V,W) \rightarrow (A,B)$ is a pair $(f : V \hookrightarrow
A,
g  : (A-f(V))\perp (f\perp id)(W) \xrightarrow{\sim} B)$.
\end{definition}

To justify this definition, we need to describe the relationship
between $\widetilde{\mathscr GW}(R)$ and $\mathscr GW(R)$.

Let $\mbb N$ denote the discrete category on the natural numbers with
its usual symmetric monoidal structure, and
let $\mbb N^{-1}\mbb N$ denote Grayson's group completion of this
symmetric monoidal category outlined above. There's a functor $\mbb
N^{-1}\mbb N \rightarrow \Z$, where $\Z$ is the discrete category on
the integers, defined on objects by $(n,m) \mapsto n-m$. This functor
is non-canonically split by the functor $\Z \rightarrow \mbb
N^{-1}\mbb N$, $z \mapsto (z,0)$, and these two functors yield weak equivalences
equivalences after application of the nerve. 

Consider the map 
\[
Fr: \mbb N^{-1}\mbb N  \rightarrow GW(R)
\]
defined on objects by
\[
(n,m) \mapsto (R^n,R^m)
\]
where $R^n,R^m$ have bilinear form corresponding to the identity
matrix and 

\begin{align*}
&R^n
\hookrightarrow \mbb H^\infty_R \perp 0\\ 
&R^m \hookrightarrow \mbb H^\infty_R \perp 0
\perp 0
\end{align*}

On morphisms an equivalence class $(k,a_0,a_1) : (n_0,n_1) \rightarrow
(m_0,m_1)$ such that $a_i : n_i + k = m_i$ is sent to the isometry
$(R^k,a_0,a_1)$ where  $a_i$ is the canonical isometry  $R^{n_i} \perp R^k \cong R^{m_i}$.

Consider as well the map 
\[
\iota: \widetilde{GW}(R) \rightarrow GW(R)
\]
 defined
on objects by
\[
(V,W) \mapsto (0 \perp V,0 \perp W)
\]
where
\begin{align*}
&0 \perp V \hookrightarrow 0 \perp \mbb H^\infty_R \\
&0 \perp W \hookrightarrow 0 \perp
V \perp \mbb H^\infty_R.
\end{align*}

For morphisms,note that given a morphism $(f,g) : (V,W) \rightarrow
(A,B)$ in $\widetilde{\mathscr GW}(R)$, there are induced isometries
\[
\widetilde{f} : A-f(V) \perp V \xrightarrow{id \perp f} A-f(V) \perp
f(V) \xrightarrow{\sim} A
\]
\[
\widetilde{g} : A-f(V) \perp W \xrightarrow{id \perp f} A-f(V) \perp (f \perp id)(W)
\xrightarrow{g} B.
\]  
Send such a pair $(f,g)$ to the triple $(A-f(V),\widetilde f,
\widetilde g) : (V,W) \rightarrow (A,B)$ in $\mathscr GW(R)$.


Now consider the composite functor
\begin{equation}\label{RankSplit}
\mbb N^{-1} \mbb N \times \widetilde{\mathscr GW}(R) \xrightarrow{Fr
  \times \iota}\mathscr GW(R) \times \mathscr GW(R)
\xrightarrow{\perp} \mathscr GW(R).
\end{equation}

\begin{lemma}
The functor \eqref{RankSplit} is an equivalence of categories over any
ring $R$ such that non-degenerate hermitian vector bundles have
constant rank. 
\end{lemma}

\begin{proof}
Consider the functor $\mathscr GW(R) \rightarrow \Z$
defined on objects by $(V,W) \mapsto
\mathrm{rk}(V)-\mathrm{rk}(W)$. By assumption, this is well-defined. Given a morphism $(C,a_0,a_1) :
(V_0,V_1) \rightarrow (W_0,W_1)$, send it to the morphism
$id_{\mathrm{rk}(W_0)}$. Now consider the commutative diagram
\[
\xymatrix{\widetilde{\mathscr GW}(R) \ar[r]^{\iota} & \mathscr GW(R)
\ar[r]^-{\mathrm{rk}} & \Z \\ \widetilde{\mathscr GW}(R) \ar[u]^{id}
\ar[r]^-{0 \times id} & \Z \times \widetilde{\mathscr GW}(R) \ar[u]^{T} \ar[r]^-{\pi_\Z} & \Z \ar[u]^{id}},
\]
where $T$ is the composite
\[
\Z \times \widetilde{\mathscr GW}(R) \rightarrow 
  \mbb N^{-1}\mbb N \times  \widetilde{\mathscr GW}(R)  \xrightarrow{\eqref{RankSplit}} \mathscr GW(R).
\]

After applying the nerve, we get a diagram of fibrations of grouplike $E_\infty$ spaces (and the
maps are maps of $E_\infty$ spaces) so that $T$ is a weak equivalence
by the 5-lemma. 
\end{proof}

\begin{corollary}
The functor \eqref{RankSplit} is a weak equivalence in the equivariant
Nisnevich topology, and hence an equivariant $\mbb A^1$-equivalence. 
\end{corollary}

\begin{proof}
The points in the equivariant Nisnevich topology have the form
\[
R = C_2 \times^{S_x} \Spec(\mc O^h_{X,x}).
\]
If $S_x = C_2$, then $R$ is a local ring and hence connected. If $S_x
= \{e\}$, then $R$ is a hyperbolic ring, and non-degenerate hermitian
vector bundles have the same rank over each connected component. 
\end{proof}

Now that we've justified the definition of $\widetilde{\mathscr
  GW}(R)$, we produce maps 
\[
\RGr_\infty \rightarrow B_{et} O \rightarrow \widetilde{\mathscr
  GW}.
\]

Recall that $P$ is the poset of non-degenerate sub-bundles of
$\mbb H^\infty$. The inclusion $P \hookrightarrow \langle \mc  S(\mbb
H^\infty),\mc S(\mbb H^\infty)\rangle$ yields a natural transformation
of functors $\mc H \rightarrow \mc I$. 

\begin{definition}
Let
\begin{align*}
\hRGr_\bullet(R) =& \hocolim_{P} \RGr_{|V|}(V_R
  \perp \mbb H^\infty_R)\\
\mathscr S_\bullet(R) =& \hocolim \mc H
\end{align*}
\end{definition}

\begin{lemma}
Let $(\mathscr P,\leq)$ be a filtered poset, and let $\mathscr F:
\mathscr P \rightarrow \Cat$ be a functor from $\mathscr P$ into the
category $\Cat$ of small categories. Then the canonical functor of categories
\[
\phi : \hocolim_{\mathscr P} \mathscr F \rightarrow \colim_{\mathscr
  P} \mathscr F
\]
is a homotopy equivalence of simplicial sets after application of the
nerve. 
\end{lemma}

\begin{proof}
The tool for proving such results is Quillen's Theorem A. To use it to
conclude that $\phi$ is a homotopy equivalence, we need to show that
$N(d \downarrow \phi)$ is contractible for any object $d \in \colim_{\mathscr
  P} \mathscr F$. By definition, the comma category $d \downarrow
\phi$ has as objects pairs 
\[
 (c \in \hocolim_{\mathscr P} \mathscr F, e \in \Hom_{\colim_{\mathscr
  P} \mathscr F}(d, \phi(c)))
\] 
and morphisms $(c,e) \rightarrow (c',e')$ are maps $t: c \rightarrow c'$
which make the square
\[
\xymatrix{d \ar[r]^e \ar[d]_{id} & \phi(c) \ar[d]^{\phi(t)} \\ d \ar[r]^{e'} & \phi(c')} 
\]
commute. Given a morphism $P \leq Q$ in $\mathscr P$, and an object $A
\in \mathscr F(P)$, denote by $A_Q$
the object
$\mathscr F(P\leq Q)(A)$. A fixed object $d \in \colim_{\mathscr
  P} \mathscr F$ is represented by a pair $[P,A]$ with $P \in \mathscr
P$ and $A \in \mathscr F(P)$. Given such a pair, we claim that there
is an equivalence of categories
\[
\psi : \colim_{P \leq Q \in \mathscr P}(id_{\hocolim_{\mathscr P} \mathscr F} \downarrow (Q,A_Q)) \cong (\phi \downarrow [P,A]).
\]

Here for $Q \leq R$, the functor $(id \downarrow (Q,A_Q)) \rightarrow
(id \downarrow (R,A_R))$ sends $t : (T,B) \rightarrow (Q,A_Q)$ to $c
\circ t : (T,B) \rightarrow (R,A_R)$ with $c : (Q,A_Q) \rightarrow
(R,A_R)$ the map in the homotopy colimit given by 
\[
(Q \leq R,id: A_R = \mathscr F(Q \leq R)(A_Q)
\rightarrow A_R).
\]

The functor $\psi$ is defined on objects by $(c,e: c\rightarrow
(Q,A_Q)) \mapsto (c,e)$. This is well-defined, because $\phi(Q,A_Q) =
[P,A]$ by definition of colimit of categories. On morphisms, a map $t:
c \rightarrow c'$ over $(Q,A_Q)$ is sent to the corresponding map $t :
c \rightarrow c'$ in $(\phi \downarrow [P,A])$.

Given such an equivalence, the colimit on the left is a filtered
colimit of categories with initial objects given by $((Q,A_Q),id)$,
and hence is a filtered colimit of contractible
categories. Because the nerve commutes with filtered colimits, and
simplicial homotopy groups commute with filtered colimits, it follows
that the comma category $(\phi \downarrow [P,A])$ is contractible just
as desired.
\end{proof}

\begin{corollary}
There are homotopy equivalences
\begin{align*}
\hRGr_\bullet &\xrightarrow{\sim} \RGr\\
\mathscr S_\bullet &\xrightarrow{\sim} \mc S_\bullet.
\end{align*}
\end{corollary}

Now, there's a sequence of maps
\[
\hRGr_\bullet \xrightarrow{\phi} \mathscr S_\bullet \xrightarrow{\psi}
\widetilde{\mathscr GW}
\]
where $\phi$ is induced by inclusion of objects and $\psi$ is induced
by inclusion of diagrams. 

We now state the theorem that will set us up for a geometric model of
$\widetilde{\mathscr GW}$ in the equivariant $\mbb A^1$-homotopy
category. 

\begin{theorem}
Let $R$ be a commutative regular Noetherian ring which is either
connected or hyperbolic, and with
$\frac{1}{2} \in R$. Then there are weak equivalences
\[
B\hRGr_\bullet(\Delta R) \xrightarrow{\phi} |B\mathscr S_\bullet(\Delta R)| \xrightarrow{\psi}
|B\widetilde{\mathscr GW}(\Delta R)|
\]
where $| - |$ denotes the diagonal of a bisimplicial set. 
\end{theorem}

\begin{remark}
The theorem is evidently false if we replace $\Delta R$ by $R$, since
$B\hRGr_\bullet(R)$ is 0-truncated. 
\end{remark}

\begin{proposition}
Let $R$ be a regular noetherian ring with involution such that
non-degenerate hermitian vector bundles over $R$ have constant rank,
and such that $\frac{1}{2} \in R$. Then inclusion of diagrams induces
a weak equivalence of simplicial sets
\[
\mathscr S_\bullet(\Delta R) \xrightarrow{\sim} \widetilde{\mathscr{G}W}(\Delta R).
\]
\end{proposition}



\begin{proof}
By the Group Completion Theorem, the map
\[
\mathscr S_\bullet( R) \xrightarrow{\sim} \widetilde{\mathscr{G}W}( R)
\]
is an isomorphism on integral homology groups. To write this out more
explicitly, we have a commutative diagram
\[
\xymatrix{\mathscr S_\bullet(R) \ar[r]\ar[d] &
  \widetilde{\mathscr{G}W}( R) \ar[d] \\
\colim_{V \subset \mbb H_R^\infty} \mc S(V \perp\mbb H_R^\infty) \ar[r]
\ar[d] & \colim_{\mc I}\mc S(V \perp\mbb H_R^\infty)  \ar[d] \ar[r]^-\sim
& GW(R)\\
\Z \ar[r] & \Z}.
\]

Using that homology commutes with filtered colimits of simplicial
sets together we compute that the homology of $\colim_{V \subset \mbb
  H_R^\infty} \mc S(V \perp\mbb H_R^\infty)$ is the group completion
of the homology of $\mc S(V \perp\mbb H_R^\infty)$, and by the group
completion theorem so is the homology of $GW(R)$.

 Said another way, the
map
\[
\Z\mathscr S_\bullet( R) \xrightarrow{\sim} \Z\widetilde{\mathscr{G}W}( R)
\]
is a weak equivalence. It follows that
\[
\Z\mathscr S_\bullet(\Delta R) \xrightarrow{\sim} \Z\widetilde{\mathscr{G}W}(\Delta R)
\]
is a level-wise weak equivalence of bisimplicial sets, and hence is a
weak equivalence after taking the diagonal. It follows that the map in
the proposition is an isomorphism on integral homology. 

Now for regular rings with involution, we have $GW(\Delta R) \cong
GW(R)$ and hence $\widetilde{\mathscr GW}(\Delta R) \cong
\widetilde{\mathscr GW}( R)$ are group complete $H$-spaces. 

Note that the $E_\infty$-structure defined above on $\mc
S_\bullet(\Delta R)$ gives an $E_\infty$-structure on $\mathscr
S_\bullet(\Delta R)$ simply by replacing all limits/colimits in the
definition with homotopy limits/colimits. Now we have a map $\mathscr
S_\bullet(\Delta R) \rightarrow \widetilde{\mathscr GW}(\Delta R)$ of
group complete $H$-spaces which is a homology isomorphism. By
uniqueness of group completions, It follows
that the map is a homotopy equivalence. 
\end{proof}

\begin{corollary}
There's an equivariant motivic equivalence $L_{mot}\RGr_\bullet
\xrightarrow{\sim} L_{mot}\widetilde{\mathscr{G}W}$.
\end{corollary}

\section{An $E_\infty$ structure on the Hermitian $K$-Theory Spectrum}

\subsection{A projective bundle formula for $\mbb P^\sigma$}


Consider the square
\[
\xymatrix{\mc O(-1)\ar[r]^{\frac{T-S}{2}}\ar[d]^{\frac{T+S}{2}} & \mc O \ar[d]^{\frac{T+S}{2}} \\ \iHom(\sigma_*\mc
  O,\mc O) \ar[r]^{\frac{T-S}{2}} & \iHom(\sigma_* \mc O(-1),\mc O)}
\]

where the map $\frac{T+S}{2} : \mc O(-1) \rightarrow \mc
O$ is induced via adjunction by the composition 
\[
\mc O(-1) \otimes \left\{\frac{T+S}{2}\right\} \otimes \sigma_* \mc O
\xrightarrow{id \otimes i \otimes id}
\mc O(-1) \otimes \mc O(1) \otimes \sigma_*\mc O  
\xrightarrow{id \otimes id \otimes (\sigma^\#)^{-1}} \mc O(-1) \otimes
\mc O(1) \otimes \mc O \xrightarrow{\mu \otimes id} \mc O \otimes \mc
O \xrightarrow{\mu} \mc O
\]
where $\mu$ denotes multiplication, and the map $\frac{T-S}{2} :
\iHom(\sigma_* \mc O, \mc O) \rightarrow \iHom(\sigma_* \mc O(-1),\mc
O)$ is induced by $\sigma_*(\frac{T-S}{2})$, which is still
multiplication by the global section $\frac{T-S}{2}$. 

Note that these are both well-defined maps of $\mc O$-modules. Denote
this form by $\phi$.

In order to show that this is a symplectic form, we have to check that
$\phi^* \circ (-\can) = \phi$. 



In \cite{Schder}, Schlichting defines a spectrum $GW(\mathscr A)$
associated to any dg category with weak equivalences and
duality. Considering the dg category with weak equivalences and
duality associated to a ring with involution $(R,\sigma)$, his results
imply that $\Z \times \widetilde{\mathscr GW}(R) \cong
\Omega^\infty GW(R)$. Together with the results of the previous
section, we see that there's an equivariant motivic equivalence $L_{mot}(\Z
\times \RGr_\bullet(R)) \cong L_{mot}\Omega^\infty GW$.

\begin{enumerate}
\item Check that $GW$ (non boldface) is a presheaf of
  $E_\infty$-spectra.

We know that $GW$ is monoidal from dg categories with weak
equivalences to spectra. 

Consider the presheaf of categories with weak equivalence and duality
on $\Sm{S}^{C_2}$ given by $f : X \rightarrow S \mapsto BigVB(S)$, or some
other strictly functorial version of the category of projective
modules. This is (strongly) symmetric monoidal via tensor product of
modules. 
(prove that you actually get symmetric monoidal structure on
dgcatwe). It follows that upon application of $GW$ we get a symmetric
monoidal presheaf of spectra. 
\item Identify $\mbb GW$ with a periodization of $GW$.

Note that in Hu-Kriz they explicitly produce a periodic spectrum.

\item Use marc's technology to deduce that $\mbb GW$ is a motivic
  $E_\infty$-ring spectrum. 
\end{enumerate}

Fix a scheme with involution $(X,\sigma)$, and let $GW$ be the
Grothendieck-Witt spectrum of $(X,\sigma)$ as defined in
\cite{Xie2018ATM}. It follows from results in \cite{Schder} that $GW((X,\sigma))$ is
a symmetric monoid in the category of symmetric spectra, and hence an
$E_\infty$-ring spectrum. It also follows from (loc. cit. )that $GW$ yields a presheaf
of symmetric monoids in symmetric spectra, and in particular an
$E_\infty$ object in the $\infty$-category of presheaves of spectra on
$\Sm{S}^{C_2}$. 

For a scheme $X$ (possibly with involution), denote by $GW^{[n]}(X)$ the
Grothendieck-Witt spectrum of the dg-category with weak equivalences
and duality given by the category of strictly perfect complexes on $X$
with duality induced by the shifted duality coefficient corresponding
to $\mc O_X[-n]$. 

Now fix a scheme $X$ with trivial involution. The projective bundle
formula in \cite{Schder} yields a natural isomorphism
$GW^{[n]}(X) \oplus GW^{[n-1]}(X) \cong GW^{[n]}(\mbb P^1_X)$, while
the Devissage theorem in \cite{Xie2018ATM} yields a natural isomorphism 
$GW^{[n]}(\mbb P^\sigma_X) \cong GW^{[n-1]}(X,-can) \oplus
GW^{[n]}(X)$, where $GW^{[n-1]}(X,-can)$ denotes symplectic hermitian
$K$-theory. Thus working with pointed versions of $\mbb P^\sigma, \mbb
P^1$, we have $GW^{[n]}(\mbb P^1_X,[1:1]) \cong GW^{[n-1]}(X)$ and
$GW^{[n]}(\mbb P_X^\sigma,[1:1]) \cong GW^{[n-1]}(X,-can)$.

Because this isomorphism is natural, it follows that, as a presheaf of spectra, $\Omega^{\mbb P^1}GW \cong
\Omega^{\mbb P^\sigma}  \prescript{}{-1}{GW}$. Using Karoubi's fundamental theorem
(\cite{Schder} Theorem 6.2),
Hu-Kriz prove that there are periodicity equivalences 
\begin{align}
\Omega^{\mbb G_m \wedge S^\sigma}GW &\cong GW \label{per:1}\\
\Omega^4 GW &\cong \Omega^{S^{4\sigma}}GW. \label{per:2}
\end{align}

Using Karoubi's fundamental theorem and its consequences, together
with the fact that loop spaces commute and we're $S^1$-stable, we get
\begin{align*}
\Omega^{\mbb P^1}\Omega^{\mbb P^\sigma}  \prescript{}{-1}{GW} &\cong
\Omega^{\mbb P^1}\Omega^{\mbb P^1} GW \\
\implies \Omega^2 \Omega^{\mbb P^1}\Omega^{\mbb P^\sigma}  \prescript{}{-1}{GW} &
  \cong \Omega^2 \Omega^{\mbb P^1}\Omega^{\mbb P^1} GW\\
\implies \Omega^{S^{2\sigma}} \Omega^{\mbb P^1}\Omega^{\mbb P^\sigma}
  GW & \cong  \Omega^2 \Omega^{\mbb P^1}\Omega^{\mbb P^1} GW\\
\implies \Omega^{S^{4\sigma}} \Omega^{\mbb P^1}\Omega^{\mbb P^\sigma}
  GW & \cong  \Omega^{S^{2\sigma}} \Omega^2 \Omega^{\mbb
       P^1}\Omega^{\mbb P^1} GW\\
\implies \Omega^4 \Omega^{\mbb P^1}\Omega^{\mbb P^\sigma}
  GW & \cong  \Omega^{S^{2\sigma}} \Omega^4 \Omega^{\mbb
       G_m^{\wedge 2}} GW\\
\implies \Omega^4 \Omega^{\mbb P^1}\Omega^{\mbb P^\sigma}
  GW & \cong \Omega^4  GW\\
\implies \Omega^{\mbb P^1}\Omega^{\mbb P^\sigma}
  GW & \cong GW.
\end{align*}

 Issue, we're using equivalences in motivic spaces. The equivalence
 $\mbb P^1 \cong S^1 \wedge \mbb G_m$ is a zig-zag of motivic spaces. 

There's an equivalence $\Omega \Omega^{\mbb G_m} GW \rightarrow U$ given by
the connecting map in the $GW$-localization sequence. Thus there's an
equivalence $\Omega^{S^\sigma} \Omega \Omega^{\mbb G_m} GW \rightarrow
\Omega^{S^\sigma} U$. Now HKO prove that there's an equivalence of fiber
sequences of presheaves of spectra
\[
\xymatrix{\Omega^{S^\sigma}U \ar[r] \ar[d] &  \Omega^{S^\sigma}F(\Z/2_+,GW)
  \ar[d] \ar[r]^-{\Omega^{S^\sigma} H} &\Omega^{S^\sigma} GW \ar[d]\\
\Omega GW \ar[r]  & F(\Z/2_+ \wedge S^\sigma,GW) \ar[r] &
\Omega^{S^\sigma}GW 
}.
\]

where the bottom sequence is obtained from applying the functor
$F(-,GW)$ to the cofiber sequence of pointed spaces given by the pinch map
\[
S^\sigma \rightarrow \Z/2_+ \wedge S^\sigma \rightarrow S^1.
\]

Composing with the above equivalences gives a map $\Omega^{S^\sigma}
\Omega \Omega^{\mbb G_m} GW \rightarrow \Omega^{S^\sigma} U
\rightarrow \Omega GW$.
\bibliography{Thesisbib}
\bibliographystyle{plain}
\end{document}